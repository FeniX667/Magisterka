\chapter{Przegląd wybranych algorytmów i innych rozwiązań}
\label{cha:rozdz3}

W tym rozdziale przedstawiam kilka wybranych algorytmów, które zaimplementuję w następnym rozdziale. Dla każdego z nich opisuję jego użycie w odniesieniu do gry 'Love Letter'.

\section{Algorytm losowy}
\label{sec:algLos}
\subsection{Opis}
Jest to najprostszy algorytm podejmowania decyzji. Na wejściu otrzymuje listę dostępnych zagrań, z których w całkowicie losowy sposób wybiera jedno. Każde z dostępnych zagrań ma takie samo prawdopodobieństwo bycia wybranym przez ten algorytm.

\subsection{Sposób wykorzystania}
Z uwagi na prostotę tego algorytmu, używam go jako punktu odniesienia dla wszystkich pozostałych strategii. Dla każdej z nich algorytm losowy spełnia rolę drugiego gracza i w ten sposób można łatwo porównać je ze sobą. Dla tego algorytmu zastosowałem jedną drobną modyfikację: nigdy nie podejmie decyzji o zagraniu Księżniczki - oznacza to natychmiastową przegraną niezależnie od momentu gry, co mogłoby mocno wypaczyć wyniki innych algorytmów.

\section{Algorytm zachłanny}
\label{sec:algZach}
\subsection{Opis}
Algorytm zachłanny polega na wyborze najlepszego możliwego zagrania dostępnego w danej chwili, nie analizując jego konsekwencji w przyszłości. Pomimo, że takie podejmowanie decyzji krótkowzroczne, jest łatwe w implementacji i daje atrakcyjne wyniki w niektórych problemach, np. przy szukaniu minimalnego drzewa rozpinającego\textsuperscript{[\ref{bib:algorytmy_zachlanny}]}.

\subsection{Sposób wykorzystania}

W kontekście gry 'Love Letter', implementacja algorytmu zachłannego wymaga pewnego doprecyzowania. Najważniejszą częścią jest funkcja kryterialna oceniająca wartość zagrania, która w niektórych przypadkach musi być oparta o probabilistykę wystąpienia kart u przeciwnika. 

Rozważmy przypadek, w którym algorytm musi podjąć decyzję o zagraniu karty Strażniczki, lub karty Barona. Jest to pierwszy ruch gracza, a w widocznych kartach odrzuconych na starcie są odrzucone karty Króla, Księcia i Pokojówki. Oznacza to, że w talii pozostało 9 kart, a jedna z odrzuconych jest niewidoczna, niemniej jednak ją też trzeba brać pod uwagę. Wyliczenie prawdopodobieństwa \textit{P} jaką kartę ma przeciwnik jest tym momencie proste, jednak trzeba jeszcze wziąć pod uwagę drobny szczegół - czy do liczenia \textit{P} wliczać karty posiadane w ręce. Z jednej strony wydaje się to nielogiczne i może prowadzić do wybierania nieoptymalnych decyzji (co przeczyłoby idei algorytmu zachłannego), z drugiej strony można to potraktować jako element blefu, który jest nieodłączną częścią każdej gry towarzyskiej. W swojej implementacji założyłem absolutną zachłanność algorytmu i karty posiadane na ręce są pomijane w obliczeniach. 
Wobec powyższego, prawdopodobieństwo wystąpienia kart u przeciwnika rozkłada się następująco:


\begin{table}[h]
	\caption{Przykład rozkładu prawdopodobieństwa}
	\centering
		\begin{tabular}{|l|r|}
			\hline
			Karta & $P_{(Karta)}$	\\ \hline
			Strażniczka & 30\% 			\\ \hline
			Kapłan & 20\% 				\\ \hline
			Baron & 10\% 				\\ \hline
			Strażniczka & 10\% 			\\ \hline
			Książę & 10\% 				\\ \hline
			Hrabina & 10\% 				\\ \hline
			Księżniczka & 10\% 			\\ \hline
		\end{tabular}
\end{table}

Zagranie karty Baron oznacza porównanie drugiej karty z kartą przeciwnika. Łatwo policzyć, że w 70\% przypadków skończyłoby to się porażką, a w 30\% nie było by żadnego efektu. Drugim dostępnym ruchem jest zagranie karty Strażniczki, a w jej przypadku najlepszym wyborem jest wytypowanie Kapłana, co daje 20\% szans na zwycięstwo i 80\% szans, że nie nastąpi żaden efekt. Zauważmy, że gdybyśmy wliczali posiadane karty do obliczenia prawdopodobieństwa, wystąpienie Barona i Kapłana byłoby tak samo możliwe. W takich przypadkach algorytm powinien zawsze celować w kartę z wyższym numerem. By formalnie stwierdzić, jaka decyzja powinna zostać podjęta, musimy obliczyć funkcję kryterialną dla dostępnych zagrań i wybrać to zagranie, dla której funkcja przyjmuje wyższy wynik. Przyjmijmy, że funkcja kryterialna wygląda następująco:

\begin{center}
	$F_{(karta)} = 1 + prawdopodobienstwo\_wygranej - prawdopodobienstwo\_przegranej$
\end{center}
Po podstawieniu otrzymujemy:
\begin{center}
 $F_{(Strazniczka)}=1.2$ i $F_{(Baron)} = 0.3$
 \\
 $F_{(Strazniczka)}>F_{(Baron)} => Decyzja=Zagranie_{(Pokojowka + Kaplan)}$ 
 \end{center} 

Najlepszą decyzją w tym wypadku jest zagranie karty Strażniczki z wyborem karty Kapłana. Jak jednak na podstawie powyższego wzoru ocenić zagranie karty Kapłana, Pokojówki lub Króla? Każda z nich wymaga unikalnej funkcji kryterialnej. Mając na uwadze, że w kontekście strategii zachłannej decyzja zawsze powinna być optymalna w ujęciu chwili, ustaliłem następujące warunki oceny zagrania karty:
\begin{itemize}
	\item Strażniczka - ocena zagrania wzrasta gdy pozwala wyeliminować przeciwnika.
	\item Kapłan - w ujęciu chwili jest to karta neutralna i ocena jej zagrania będzie zawsze stała.
	\item Baron - ocena zagrania wzrasta gdy mamy drugą kartę silniejszą niż może mieć przeciwnik.
	\item Pokojówka - podobnie jak w przypadku kapłana, ocena zagrania karty będzie stała.
	\item Książę - ocena wzrasta z prawdopodobieństwem Księżniczki u przeciwnika, lub gdy druga posiadana karta ma mniejszą siłę niż może mieć przeciwnik (wtedy zagrywana na siebie)
	\item Król - premiowane gdy przeciwnik ma Księżniczkę lub Hrabinę.
	\item Hrabina - karta neutralna, jednak musimy ją zagrać w określonych sytuacjach.
	\item Księżniczka - nie może być nigdy wyrzucona.
	\item Dodatkowo, jeśli oba zagrania mają taką samą wartość, powinna być podjęta decyzja o zagraniu karty o niższej wartości.
\end{itemize}

\subsection{Zapis pseudokodem}
Strategię zachłanną $D_n$ najlepiej przedstawić za pomocą pseudokodu:
\begin{algorithmic}[1]
	\Function{$D_n$}{$X_n,s_{n-1}$}
		\ForAll{ rodzajKarty } \Comment Tworzenie tablicy prawdopodobieństwa
			\State $P[rodzajKarty] \gets$  prawdopodobieństwo wystąpienia karty danego rodzaju u przeciwnika	
		\EndFor
		\ForAll{ $x_i$ } \Comment Obliczenie funkcji kryterialnej dla każdego zagrania
				\State K[$x_i$] $\gets$ $F(x_i, P[])$
		\EndFor
		
		\State $ decyzja \gets x_0$ \Comment Szukanie zagrania o najwyższej wycenie
		\ForAll{ $x_i$ } 
			\If {$K[decyzja] < K[x_i]$}
				\State $decyzja \gets x_i$
			\EndIf
		\EndFor

		\State \textbf{return} $decyzja$
	\EndFunction
\end{algorithmic}

Gdzie funkcja kryterialna wygląda następująco:
\begin{algorithmic}[1]
	\Function{$F$}{$x_i,P[]$}	
		\Switch{$x_i$}
			\Case{Strażniczka + Kaplan} \Comment Zagranie Strażniczki na dany typ karty
				\State \textbf{return} $ 1 + P[Kaplan] + 0.08 $
			\EndCase
			\Case{Strażniczka + Baron}
				\State \textbf{return} $ 1 + P[Baron]  + 0.08 $
			\EndCase
			\State ...
			\Case{Strażniczka + Księżniczka}
				\State \textbf{return} $ 1 + P[Ksiezniczka]  + 0.08 $
			\EndCase
			\Case{Kapłan}
				\State \textbf{return} $ 1 + 0.07 $
			\EndCase
			\Case{Baron}	\Comment Kryterium zależy od wartości W() drugiej posiadanej karty (DK)
				\State $ szansePrzegranej \gets 0$ 
				\State $ szanseWygranej \gets 0$ 
				\ForAll {$typKarty$}
					\If {$ W(DK) < W(typKarty) $}  
						\State $szansePrzegranej \gets szansePrzegranej + P[typKarty]$ 
					\ElsIf {$ W(DK) > W(typKarty) $}
					\State $szanseWygranej \gets szanseWygranej + P[typKarty]$ 
					\EndIf
				\EndFor
				\State \textbf{return} $ 1 - szansePrzegranej + szanseWygranej + 0.06 $
			\EndCase
			\Case{Pokojówka}
				\State \textbf{return} $ 1 + 0.05 $
			\EndCase
			\Case{Książe na przeciwnika}
				\State \textbf{return} $ 1 + P[Ksiezniczka] + 0.04 $
			\EndCase
			\Case{Książe na siebie} 
				\State \textbf{return} $ 1 + 0.04 $
			\EndCase
			\Case{Król}
				\State \textbf{return} $ 1 + P[Ksiezniczka] + P[Hrabina] + 0.03 $
			\EndCase
			\Case{Hrabina, gdy druga posiadana karta to Król lub Książe}
				\State \textbf{return} $ 10 + 0.02 $
			\EndCase
			\Case{Hrabina}
				\State \textbf{return} $ 1 + 0.02 $
			\EndCase
			\Case{Księżniczka}
				\State \textbf{return} $ 0 $
			\EndCase
		\EndSwitch
	\EndFunction
\end{algorithmic}

		
\section{Algorytm Monte Carlo Tree Search}
\label{sec:algMCTS}