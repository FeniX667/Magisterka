\chapter{Przegląd wybranych algorytmów i innych rozwiązań}
\label{cha:rozdz3}

\section{Wstęp}
\label{sec:wstep2}
W tym rozdziale przedstawiam kilka wybranych algorytmów oraz metod, które zaimplementuję w następnym rozdziale. Dla każdej z nich opisuję ich użycie w odniesieniu do gry 'Love Letter'.

\section{Algorytm losowy}
\label{sec:algLos}
\subsection{Opis}
Jest to najprostszy algorytm podejmowania decyzji. Na wejściu otrzymuje listę dostępnych decyzji, z których w sposób całkowicie losowy podejmuje jedną. Każda z dostępnych decyzji ma takie samo prawdopodobieństwo bycia wybraną przez ten algorytm.

\subsection{Sposób wykorzystania}
Z uwagi na prostotę tego algorytmu, używam go jako punktu odniesienia dla wszystkich pozostałych strategii. Dla każdej z nich algorytm losowy spełnia rolę drugiego gracza i w ten sposób można łatwo porównać je ze sobą. Dla tego algorytmu zastosowałem jedną drobną modyfikację: nigdy nie podejmie decyzji o zagraniu Księżniczki - oznacza to natychmiastową przegraną niezależnie od momentu gry, co mogłoby mocno wypaczyć wyniki innych algorytmów.

\section{Algorytm zachłanny}
\label{sec:algZach}
\subsection{Opis}
Algorytm zachłanny polega na podjęciu najlepszej możliwej decyzji dostępnej w danej chwili, nie analizując ich konsekwencji w przyszłości\textsuperscript{[\ref{bib:algorytmy_zachlanny}]}. Pomimo, że jest to działanie krótkowzroczne, jest łatwe do implementacji i daje atrakcyjne wyniki w niektórych problemach, np. przy szukaniu minimalnego drzewa rozpinającego.

\subsection{Sposób wykorzystania}

W kontekście gry 'Love Letter', implementacja algorytmu zachłannego wymaga pewnego doprecyzowania. Najważniejszą częścią jest funkcja oceny decyzji, która w niektórych przypadkach musi być oparta o probabilistykę wystąpienia kart u przeciwnika. 

Rozważmy scenariusz, w którym algorytm musi podjąć decyzję zagrania karty Strażniczki, lub karty Barona. Jest to pierwszy ruch gracza, a w widocznych kartach odrzuconych na starcie są odrzucone karty Króla, Księcia i Pokojówki. Oznacza to, że w talii pozostało 9 kart, a jedna z odrzuconych jest niewidoczna, niemniej jednak też ją trzeba brać pod uwagę. Wyliczenie prawdopodobieństwa \textit{P} jaką kartę ma przeciwnik jest tym momencie proste, jednak trzeba jeszcze wziąć pod uwagę drobny szczegół - czy do liczenia \textit{P} wliczać karty posiadane w ręce. Z jednej strony wydaje się to nielogiczne i może prowadzić do wybierania nieoptymalnych decyzji (co przeczyłoby idei algorytmu zachłannego), z drugiej strony można to potraktować jako element blefu, który jest nieodłączną częścią każdej gry towarzyskiej. Z tego powodu zaimplementowane będą dwie wersje: klasyczna i blefująca.

Przyjmując do obliczeń wersję klasyczną, prawdopodobieństwo wystąpienia kart u przeciwnika rozkłada się następująco:


\begin{center}
	Tutaj będzie tabelka zamiast na następnej stronie. %TODO 
\end{center}

\clearpage
\begin{table}[t]
	\caption{Przykład rozkładu prawdopodobieństwa}
	\centering
		\begin{tabular}{|l|r|}
			\hline
			Karta & $P_{(karta)}$	\\ \hline
			Strażniczka & 30\% 			\\ \hline
			Kapłan & 20\% 				\\ \hline
			Baron & 10\% 				\\ \hline
			Strażniczka & 10\% 			\\ \hline
			Książę & 10\% 				\\ \hline
			Hrabina & 10\% 				\\ \hline
			Księżniczka & 10\% 			\\ \hline
		\end{tabular}
\end{table}

Zagranie Barona oznacza porównanie drugiej karty z kartą przeciwnika. Łatwo policzyć, że w 70\% przypadków skończyłoby to się porażką, a w 30\% remisem. Druga dostępna decyzja to zagranie karty Strażniczki, a w jej przypadku najlepszym wyborem jest wytypowanie Kapłana, co daje 20\% szans na zwycięstwo i 80\% szans na remis. Zauważmy, że gdybyśmy wliczali posiadane karty do obliczenia prawdopodobieństwa, wystąpienie Barona i Kapłana byłoby tak samo możliwe. W takich przypadkach algorytm powinien celować zawsze w kartę z wyższym numerem. By formalnie stwierdzić która decyzja jest lepsza, przyjmijmy, że funkcja oceny wygląda następująco:

\begin{center}
	$F_{(karta)} = 1 + prawdopodobienstwo\_wygranej - prawdopodobienstwo\_przegranej$
\end{center}

Zatem $Max(F_{(Strazniczka)}, F_{(Baron)}) = 1.2$. Najlepszą decyzją w tym wypadku jest zagranie strażniczki. Jak jednak na podstawie powyższego wzoru określić działanie Kapłana, Pokojówki lub Króla? Każda z tych kart wymaga unikalnej funkcji oceny. Mając na uwadze, że decyzja zawsze powinna być optymalna w ujęciu danego momentu, ustaliłem następujące warunki oceny zagrania karty:
\begin{itemize}
	\item Strażniczka - zagranie premiowane gdy pozwala wyeliminować przeciwnika.
	\item Kapłan - jest to karta neutralna, jednak jej przydatność wzrasta jeśli nasza druga karta to Strazniczka lub Ksiaze.
	\item Baron - premiowane gdy mamy drugą kartę silniejszą niż może mieć przeciwnik
	\item Pokojówka - im silniejszą posiadamy drugą kartę, tym bardzo warto ją zagrać
	\item Książę - ocena wzrasta z prawdopodobieństwem Księżniczki u przeciwnika, lub gdy druga karta ma mniejszą siłę niż może mieć przeciwnik (wtedy zagrywana na siebie)
	\item Król - premiowane gdy przeciwnik ma Księżniczkę lub Hrabinę
	\item Hrabina - karta neutralna, jednak musimy ją zagrać w określonych sytuacjach
	\item Księżniczka - nie powinna być nigdy wyrzucona.
\end{itemize}

\subsection{Zapis formalny}
\textbf{Strategia zachłanna:}

Wejście - (obecny stan gry, karta dociągnięta)

Wyjście - podjęta decyzja
\begin{enumerate}
	\item Stwórz tablicę prawdopodobieństwa.
	\item Dla obu posiadanych kart oblicz $F_{(Karta)}$.
	\item Zwróć decyzję o najwyższej ocenie. Jeśli obie karty mają taką samą ocenę, zagraj tę o niższym numerze.
\end{enumerate}

\textbf{Funkcja oceny:}

Skróty: DK - druga karta, $S_{(n)}$ - siła karty
\begin{itemize}
	\item $F_{(Strazniczka)} = 1 + Max_{(P_{(Zbior\_kart - \{Strazniczka\})})}$
	\item $F_{(Kaplan)} = 1 + 0.25 * (DK==Ksiaze \lor DK==Strazniczka) $
	\item $F_{(Baron)} = 1 + \sum P_{Zbior\_kart\_o\_mniejszej\_sile} - \sum P_{Zbior\_kart\_o\_wiekszej\_sile} $
	\item $F_{(Pokojowka)} = 1 + 0.25 * (S_{(DK)} - S_{(Pokojowka)} )$
	\item $F_{(Ksiaze)} = 1 + P_{(Ksiezniczka)} + 0.1*(S_{DK}<5)$ 
	\item $F_{(Krol)} = 1 + P_{(\{Ksiezniczka, Hrabina\})}$
	\item $ F_{(Hrabina)} = \left\{ 
								\begin{array}{ll}
									10 & \mbox{if $DK = Ksiaze$}; \\
									10 & \mbox{if $DK = Krol$}; \\
									1 & $w innym wypadku$.
								\end{array} 
							\right.$	
	\item $F_{(Ksiezniczka)} = 0$
\end{itemize}
		
\section{Algorytm Monte Carlo Tree Search}
\label{sec:algMCTS}