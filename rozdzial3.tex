\chapter{Przegląd wybranych algorytmów i innych rozwiązań}
\label{cha:rozdz3}

W tym rozdziale przedstawiam kilka wybranych algorytmów, które zaimplementuję w następnym rozdziale. Dla każdego z nich opisuję jego użycie w odniesieniu do gry 'Love Letter'.

\section{Algorytm losowy}
\label{sec:algLos}
\subsection{Opis}
Jest to najprostszy algorytm podejmowania decyzji. Na wejściu otrzymuje listę dostępnych zagrań, z których w całkowicie losowy sposób wybiera jedno. Każde z dostępnych zagrań ma takie samo prawdopodobieństwo bycia wybranym przez ten algorytm.

\subsection{Sposób wykorzystania}
Z uwagi na prostotę tego algorytmu, używam go jako punktu odniesienia dla wszystkich pozostałych strategii. Dla każdej z nich algorytm losowy spełnia rolę drugiego gracza i w ten sposób można łatwo porównać je ze sobą. Dla tego algorytmu zastosowałem jedną drobną modyfikację: nigdy nie podejmie decyzji o zagraniu Księżniczki - oznacza to natychmiastową przegraną niezależnie od momentu gry, co mogłoby mocno wypaczyć wyniki innych algorytmów.

\section{Algorytm zachłanny}
\label{sec:algZach}
\subsection{Opis}
Algorytm zachłanny polega na wyborze najlepszego możliwego zagrania dostępnego w danej chwili, nie analizując jego konsekwencji w przyszłości. Pomimo, że takie podejmowanie decyzji krótkowzroczne, jest łatwe w implementacji i daje atrakcyjne wyniki w niektórych problemach, np. przy szukaniu minimalnego drzewa rozpinającego\textsuperscript{[\ref{bib:algorytmy_zachlanny}]}.

\subsection{Sposób wykorzystania}

W kontekście gry 'Love Letter', implementacja algorytmu zachłannego wymaga pewnego doprecyzowania. Najważniejszą częścią jest funkcja kryterialna oceniająca wartość zagrania, która w niektórych przypadkach musi być oparta o probabilistykę wystąpienia kart u przeciwnika. 

Rozważmy przypadek, w którym algorytm musi podjąć decyzję o zagraniu karty Strażniczki, lub karty Barona. Jest to pierwszy ruch gracza, a w widocznych kartach odrzuconych na starcie są odrzucone karty Króla, Księcia i Pokojówki. Oznacza to, że w talii pozostało 9 kart, a jedna z odrzuconych jest niewidoczna, niemniej jednak ją też trzeba brać pod uwagę. Wyliczenie prawdopodobieństwa \textit{P} jaką kartę ma przeciwnik jest tym momencie proste, jednak trzeba jeszcze wziąć pod uwagę drobny szczegół - czy do liczenia \textit{P} wliczać karty posiadane w ręce. Z jednej strony wydaje się to nielogiczne i może prowadzić do wybierania nieoptymalnych decyzji (co przeczyłoby idei algorytmu zachłannego), z drugiej strony można to potraktować jako element blefu, który jest nieodłączną częścią każdej gry towarzyskiej. Z tego powodu zaimplementowałem dwie wersje, nazwane odpowiednio klasyczną i blefującą.

Przyjmując do obliczeń wersję klasyczną, prawdopodobieństwo wystąpienia kart u przeciwnika rozkłada się następująco:


\begin{center}
	Tutaj będzie tabelka zamiast na następnej stronie. %TODO 
\end{center}

\clearpage
\begin{table}[t]
	\caption{Przykład rozkładu prawdopodobieństwa}
	\centering
		\begin{tabular}{|l|r|}
			\hline
			Karta & $P_{(Karta)}$	\\ \hline
			Strażniczka & 30\% 			\\ \hline
			Kapłan & 20\% 				\\ \hline
			Baron & 10\% 				\\ \hline
			Strażniczka & 10\% 			\\ \hline
			Książę & 10\% 				\\ \hline
			Hrabina & 10\% 				\\ \hline
			Księżniczka & 10\% 			\\ \hline
		\end{tabular}
\end{table}

Zagranie karty Baron oznacza porównanie drugiej karty z kartą przeciwnika. Łatwo policzyć, że w 70\% przypadków skończyłoby to się porażką, a w 30\% nie było by żadnego efektu. Drugim dostępnym ruchem jest zagranie karty Strażniczki, a w jej przypadku najlepszym wyborem jest wytypowanie Kapłana, co daje 20\% szans na zwycięstwo i 80\% szans, że nie nastąpi żaden efekt. Zauważmy, że gdybyśmy wliczali posiadane karty do obliczenia prawdopodobieństwa, wystąpienie Barona i Kapłana byłoby tak samo możliwe. W takich przypadkach algorytm powinien zawsze celować w kartę z wyższym numerem. By formalnie stwierdzić, jaka decyzja powinna zostać podjęta, musimy obliczyć funkcję kryterialną dla dostępnych zagrań i wybrać to zagranie, dla której funkcja przyjmuje wyższy wynik. Przyjmijmy, że funkcja kryterialna wygląda następująco:

\begin{center}
	$F_{(karta)} = 1 + prawdopodobienstwo\_wygranej - prawdopodobienstwo\_przegranej$
\end{center}
Po podstawieniu otrzymujemy:
\begin{center}
 $F_{(Strazniczka)}=1.2$ i $F_{(Baron)} = 0.3$
 \\
 $F_{(Strazniczka)}>F_{(Baron)} => Decyzja=Zagranie_{(Pokojowka, Kaplan)}$ 
 \end{center} 

Najlepszą decyzją w tym wypadku jest zagranie karty Strażniczki z wyborem karty Kapłana. Jak jednak na podstawie powyższego wzoru ocenić zagranie karty Kapłana, Pokojówki lub Króla? Każda z nich wymaga unikalnej funkcji kryterialnej. Mając na uwadze, że w kontekście strategii zachłannej decyzja zawsze powinna być optymalna w ujęciu chwili, ustaliłem następujące warunki oceny zagrania karty:
\begin{itemize}
	\item Strażniczka - zagranie premiowane gdy pozwala wyeliminować przeciwnika.
	\item Kapłan - w ujęciu chwili jest to karta neutralna, jednak jej przydatność wzrasta jeśli nasza druga karta to Strażniczka lub Książe.
	\item Baron - premiowane gdy mamy drugą kartę silniejszą niż może mieć przeciwnik.
	\item Pokojówka - im silniejszą posiadamy drugą kartę, tym bardziej warto ją zagrać.
	\item Książę - ocena wzrasta z prawdopodobieństwem Księżniczki u przeciwnika, lub gdy druga posiadana karta ma mniejszą siłę niż może mieć przeciwnik (wtedy zagrywana na siebie)
	\item Król - premiowane gdy przeciwnik ma Księżniczkę lub Hrabinę.
	\item Hrabina - karta neutralna, jednak musimy ją zagrać w określonych sytuacjach.
	\item Księżniczka - nie powinna być nigdy wyrzucona.
	\item Dodatkowo, jeśli oba zagrania mają taką samą wartość, powinna być podjęta decyzja o zagraniu karty o niższej wartości.
\end{itemize}

\subsection{Zapis formalny}
\textbf{Strategia zachłanna:}

Wejście - obecny stan gry

Wyjście - podjęta decyzja
\begin{enumerate}
	\item Stwórz tablicę prawdopodobieństwa $P_{(Karta)}$ wystąpienia karty u przeciwnika.
	\item Dla obu posiadanych kart oblicz wartość funkcji kryterialnej $F_{(Karta)}$.
	\item Jako decyzję zwróć zagranie o najwyższej ocenie.
\end{enumerate}

\textbf{Funkcja kryterialna $F_{(Karta)}$:}

Skróty: DK - druga karta w ręce, $W_{(n)}$ - wartość karty
\begin{itemize}
	\item $F_{(Strazniczka)} = 1 + Max_{(P_{(Zbior\_kart - \{Strazniczka\})})} +0.08$
	\item $F_{(Kaplan)} = 1 + 0.25 * (DK==Ksiaze \lor DK==Strazniczka) +0.07$
	\item $F_{(Baron)} = 1 + \sum P_{Zbior\_kart\_o\_mniejszej\_sile} - \sum P_{Zbior\_kart\_o\_wiekszej\_sile} +0.06$
	\item $F_{(Pokojowka)} = 1 + 0.25 * (W_{(DK)} - W_{(Pokojowka)} )+0.05$
	\item $F_{(Ksiaze)} = 1 + P_{(Ksiezniczka)} + 0.1*(W_{DK}<5)+0.04$ 
	\item $F_{(Krol)} = 1 + P_{(\{Ksiezniczka, Hrabina\})}+0.03$
	\item $ F_{(Hrabina)} = \left\{ 
								\begin{array}{ll}
									10 & \mbox{if $DK = Ksiaze$}; \\
									10 & \mbox{if $DK = Krol$}; \\
									1.02 & $w innym wypadku$.
								\end{array} 
							\right.$	
	\item $F_{(Ksiezniczka)} = 0$
\end{itemize}
		
\section{Algorytm Monte Carlo Tree Search}
\label{sec:algMCTS}