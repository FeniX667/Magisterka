\chapter{Przegląd wybranych algorytmów i innych rozwiązań}
\label{cha:rozdz3}

W tym rozdziale przedstawiam algorytmy definiujące funkcję wyboru decyzji. Są to kolejno: losowy, zachłanny, minimaksowy oraz Monte Carlo Tree Search. Dla każdego z nich opisuję jego użycie w odniesieniu do gry 'Love Letter'. Ponieważ wszystkie algorytmy będą implementowane w aplikacji komputerowej, przedstawiłem je również w postaci pseudokodu.

\section{Algorytm losowy}
\label{sec:algLos}
\subsection{Opis}
Jest to najprostszy algorytm podejmowania decyzji. Na wejściu otrzymuje listę dostępnych decyzji, z których w całkowicie losowy sposób wybiera jedną. Każda z dostępnych decyzji ma takie samo prawdopodobieństwo wyboru przez ten algorytm.

\subsection{Sposób wykorzystania}
Z uwagi na prostotę algorytmu jest on wykorzystany głównie do przetestowania poprawności działania implementacji gry. Zastosowałem w nim jedną drobną modyfikację: nigdy nie podejmie decyzji o zagraniu Księżniczki - oznacza to natychmiastową przegraną niezależnie od momentu gry, co wypaczałoby wyniki porównań algorytmów.

\subsection{Zapis pseudokodem}
\begin{algorithmic}[1]
	\Function{$d_{losowa}$}{$I_n, X_n$}	
		\ForAll{ $z_i \in X_n$ } \Comment $i =1..|X_n|$
			\If {$z_i ==$ Księżniczka}
				\State $X_n \gets X_n - z_i$
			\EndIf
		\EndFor
	\State \textbf{return} wylosujJeden($X_n$) 
	\EndFunction
\end{algorithmic}

\section{Algorytm zachłanny}
\label{sec:algZach}
\subsection{Opis}
Algorytm zachłanny polega na wyborze najlepszego możliwego zagrania dostępnego w danej chwili, nie analizując jego konsekwencji w przyszłości. Pomimo, że takie podejmowanie decyzji jest krótkowzroczne, to jest też łatwe w implementacji i daje atrakcyjne wyniki w niektórych problemach, np. przy szukaniu minimalnego drzewa rozpinającego\textsuperscript{[\ref{bib:algorytmy_zachlanny}]}.

\subsection{Sposób wykorzystania}

W kontekście gry 'Love Letter', implementacja algorytmu zachłannego wymaga pewnego doprecyzowania. Najważniejszą częścią jest funkcja kryterialna oceniająca wartość zagrania, która w niektórych przypadkach (na przykład gdy zagranie może zakończyć grę) musi być oparta na prawdopodobieństwie wystąpienia kart u przeciwnika. Prawdopodobieństwo to uzależnione jest od stanu początkowego danej tury.

Rozważmy przypadek, w którym algorytm musi podjąć decyzję o zagraniu karty Strażniczki, lub karty Barona. Jest to pierwszy ruch gracza, a w widocznych kartach odrzuconych na starcie są odrzucone karty Króla, Księcia i Pokojówki. Oznacza to, że w talii pozostało 9 kart, a jedna z odrzuconych jest niewidoczna, niemniej jednak ją też trzeba brać pod uwagę. Wyliczenie prawdopodobieństwa \textit{P[Karta]} jaką kartę ma przeciwnik jest tym momencie proste, jednak trzeba jeszcze wziąć pod uwagę drobny szczegół - czy do liczenia \textit{P[Karta]} wliczać karty posiadane w ręce. Z jednej strony wydaje się to nielogiczne i może prowadzić do wybierania nieoptymalnych decyzji (co przeczyłoby idei algorytmu zachłannego) w danym stanie, z drugiej strony można to potraktować jako element blefu, który jest nieodłączną częścią każdej gry towarzyskiej. W swojej implementacji założyłem absolutną zachłanność algorytmu i karty posiadane na ręce są pomijane w obliczeniach. 
Wobec powyższego, prawdopodobieństwa wystąpienia kart u przeciwnika wynikające z ustalonego stanu początkowego są następujące:

\begin{table}[h]
	\caption{Przykład rozkładu prawdopodobieństwa w przypadku odrzucenia kart Króla, Księcia i Pokojówki}
	\centering
		\begin{tabular}{|l|r|}
			\hline
			\bf{Karta} & $P[Karta]$	\\ \hline
			Strażniczka & 30\% 			\\ \hline
			Kapłan & 20\% 				\\ \hline
			Baron & 10\% 				\\ \hline
			Pokojówka & 10\% 			\\ \hline
			Książę & 10\% 				\\ \hline
			Hrabina & 10\% 				\\ \hline
			Księżniczka & 10\% 			\\ \hline
		\end{tabular}
\end{table}

Zagranie karty Baron oznacza porównanie drugiej karty z kartą przeciwnika. Łatwo policzyć, że w 70\% przypadków skończyłoby to się porażką, a w 30\% nie było by żadnego efektu. Drugim dostępnym ruchem jest zagranie karty Strażniczki, a w jej przypadku najlepszym wyborem jest wytypowanie Kapłana, co daje 20\% szans na zwycięstwo i 80\% szans, że nie nastąpi żaden efekt. Zauważmy, że gdybyśmy wliczali posiadane karty do obliczenia prawdopodobieństwa, wystąpienie Barona i Kapłana byłoby tak samo możliwe. W takich przypadkach algorytm powinien zawsze celować w kartę z wyższym numerem. By formalnie stwierdzić, jaka decyzja $z$ powinna zostać podjęta, musimy obliczyć funkcję kryterialną dla dostępnych zagrań i wybrać to zagranie, dla której funkcja przyjmuje wyższą wartość. Przyjmijmy (póki co na podstawie własnej intuicji), że lokalna funkcja kryterialna wygląda następująco:

\begin{center}
	$F(z) = 1 + prawdopodobienstwo\_wygranej - prawdopodobienstwo\_przegranej$
\end{center}
Po podstawieniu otrzymujemy:
\begin{center}
 $F(Str\_Kap)=1.2$ i $F(Bar) = 0.3$
 
 $F(Str\_Kap)>F(Bar) => z = Str\_Kap$ 
\end{center} 

Najlepszą decyzją w tym wypadku jest zagranie karty Strażniczki z wyborem karty Kapłana. Jak jednak na podstawie powyższego wzoru ocenić zagranie karty Kapłana, Pokojówki lub Króla? Każda z nich wymaga indywidualnej oceny. Mając na uwadze, że w kontekście strategii zachłannej decyzja zawsze powinna być najlepsza w danym stanie, ustaliłem następujące warunki którymi się kierowałem przy tworzeniu funkcji kryterialnej dla zagrań każdej z kart:
\begin{itemize}
	\item Strażniczka - ocena zagrania(wartość funkcji kryterialnej) wzrasta gdy pozwala wyeliminować przeciwnika.
	\item Kapłan - efekt karty jest neutralny i ocena jego zagrania będzie zawsze stała.
	\item Baron - ocena zagrania wzrasta gdy mamy drugą kartę silniejszą niż może mieć przeciwnik i maleje gdy jest odwrotnie.
	\item Pokojówka - podobnie jak w przypadku kapłana, ocena zagrania karty będzie stała.
	\item Książę - ocena zagrania na przeciwnika wzrasta z prawdopodobieństwem Księżniczki u przeciwnika. Ocena zagrania na siebie jest stała, lecz 0 gdy druga karta to Księżniczka.
	\item Król - jak w przypadku Kapłana i Pokojówki, ocena zagrania jest stała.
	\item Hrabina - ocena stała, jednak musimy ją zagrać, gdy druga posiadana karta to Król lub Książę.
	\item Księżniczka - nie może być nigdy wyrzucona.
	\item Dodatkowo, jeśli oba zagrania mają taką samą wartość, powinna być podjęta decyzja o zagraniu karty o niższej wartości $W$, w związku z tym ocena zagrania spada wraz z wyższą wartością karty.
\end{itemize}
Opierając się na powyższych wytycznych, zapisałem algorytm w formie pseudokodu.
\subsection{Zapis pseudokodem}
Wykorzystane zmienne pomocnicze:
\begin{itemize}
	\item $P[]$ - tablica prawdopodobieństwa wystąpień kart u przeciwnika. ($P[Karta]$ oznacza prawdopodobieństwo dla konkretnej karty)
	\item $decyzja$ - zagranie, które ma zostać zwrócone
\end{itemize}
\begin{algorithmic}[1]
	\Function{$d_{zachlanna}$}{$I_n, X_n$}
		\State $P[] \gets$ obliczPrawdopodobienstwo()
		\State $ decyzja \gets NULL$ \Comment Szukanie zagrania o najwyższej wycenie
		\ForAll{ $z_i \in X_n$ } \Comment $i=1..|X_n|$
				\If {$F(decyzja, P[]) < F(z_i, P[]$}
					\State $decyzja \gets z_i$
				\EndIf
		\EndFor		
		\State \textbf{return} $decyzja$
	\EndFunction
\end{algorithmic}

Gdzie funkcja kryterialna wygląda następująco:
\begin{algorithmic}[1]
	\Function{$F$}{$z,P[]$}	
		\Switch{$z$}
			\Case{Str\_Kap} \Comment Zagranie Strażniczki na dany typ karty
				\State \textbf{return} $ 1 + P[Kaplan] + 0.008 $
			\EndCase
			\Case{Str\_Bar}
				\State \textbf{return} $ 1 + P[Baron]  + 0.008 $
			\EndCase
			\State ...
			\Case{Str\_K-a}
				\State \textbf{return} $ 1 + P[Ksiezniczka]  + 0.008 $
			\EndCase
			\Case{Kap\_Z}
				\State \textbf{return} $ 1 + 0.007 $
			\EndCase
			\Case{Bar\_Z}	\Comment Kryterium zależy od wartości $W(DK)$
				\State $ szansePrzegranej \gets 0$ 
				\State $ szanseWygranej \gets 0$ 
				\ForAll {$typKarty$}
					\If {$ W(DK) < W(typKarty) $}  
						\State $szansePrzegranej \gets szansePrzegranej + P[typKarty]$ 
					\ElsIf {$ W(DK) > W(typKarty) $}
						\State $szanseWygranej \gets szanseWygranej + P[typKarty]$ 
					\EndIf
				\EndFor
				\State \textbf{return} $ 1 - szansePrzegranej + szanseWygranej + 0.006 $
			\EndCase
			\Case{Pok\_Z}
				\State \textbf{return} $ 1 + 0.005 $
			\EndCase
			\Case{K-e\_S}
				\State \textbf{return} $ 1 + P[\textit{K-a}] + 0.004 $
			\EndCase
			\Case{K-e\_P} 
				\If {$ DK == $ K-a}  
					\State \textbf{return} $ 0 $
				\Else
					\State \textbf{return} $ 1 + 0.004 $
				\EndIf
			\EndCase
			\Case{Krl\_Z}
				\State \textbf{return} $ 1 + P[\textit{K-a}] + P[Hra] + 0.003 $
			\EndCase
			\Case{Hra\_K-e\_Kr}
				\State \textbf{return} $ 10 + 0.002 $
			\EndCase
			\Case{Hra\_Z}
				\State \textbf{return} $ 1 + 0.002 $
			\EndCase
			\Case{K-a\_Z}
				\State \textbf{return} $ 0 $
			\EndCase
		\EndSwitch
	\EndFunction
\end{algorithmic}


\section{Algorytm min-max}
\label{sec:minmax}
\subsection{Opis}
Algorytm minimaksowy polega na ,,minimalizowaniu maksymalnych możliwych strat'' bądź alternatywnie na ,,maksymalizacji minimalnego zysku (wypłaty)''$^{[\ref{bib:wiki_minMax}]}$. Zgodnie z [\ref{bib:wazniak_minMax}] często stosuje się go do gier o następujących zasadach:
\begin{itemize}
	\item występuje dwóch graczy
	\item ruchy wykonywane są naprzemiennie
	\item w każdym stanie istnieje skończona liczba decyzji do podjęcia
	\item stan i podjęta decyzja jednoznacznie wyznaczają stan następny
	\item każdy stan może zakwalifikować do jednej z następujących kategorii:
	\begin{itemize}
		\item wygrana pierwszego gracza
		\item wygrana drugiego gracza
		\item remis
		\item sytuacja nierozstrzygnięta
	\end{itemize}
\end{itemize}
Najczęstsze implementacje polegają na przeszukiwaniu drzewa dalszych przebiegów gry począwszy od zadanego stanu początkowego, tak jak na przykład w warcabach$^{[\ref{bib:minMax_warcaby}]}$. Istnieją także implementacje opierające się na liczbowej ocenie ruchu$^{[\ref{bib:wazniak_minMax}]}$ i na tej idei opieram swoją implementację algorytmu minimaksowego.

\subsection{Sposób wykorzystania}
W podanym powyżej opisie, gra Love Letter niezgodna jest w punkcie czwartym. Jak ustaliliśmy wcześniej, gracz nie wie w jakim dokładnie stanie znajduje się gra, zna natomiast zbiór informacyjny $I_n$. Z tego względu, z perspektywy gracza $P_1$ podjęcie decyzji $z_i \in X_n$ nie określa jednoznacznie następnego węzła w drzewie, lecz loterię. Mimo to, jesteśmy w stanie statystycznie ocenić możliwości przeciwnika, a tym samym zminimalizować naszą maksymalną stratę. Ponieważ zysk należy rozumieć jako zwiększenie szansy na wygraną (tak jak w strategii zachłannej), to strata oznacza zwiększenie szansy na przegraną. 

Rozważmy scenariusz, w którym gracz pierwszy posiada kartę Księcia oraz Pokojówki. Na stosie pozostało 2 karty, 1 karta jest zakryta zgodnie z zasadami rozpoczęcia rundy i przeciwnik również posiada 1 kartę. Wśród tych 4 nieznanych graczowi kart są karty Strażniczki, Barona, Księcia oraz Księżniczki. Prawdopodobieństwo wystąpienia każdej z nich u przeciwnika wynosi 25\% i razem stanowią one tablicę prawdopodobieństwa $P[]$. Przyjmijmy, że wykorzystywana w poprzedniej strategii funkcja kryterialna $F(z, P[])$ to maksymalizacja zysku i nazwijmy ją $F_{max}(z, P[])$  Zgodnie z jej definicją $F_{max}(\textit{K-e}, P[]) = 1.29$ i $F_{max}(Pokojowka, P[]) = 1.05$, więc zagranie Księcia maksymalizuje szansę na wygraną. Jeśli jednak weźmiemy pod uwagę, że w 75\% przypadków nie wygrywamy, wówczas musimy rozważyć odpowiedź przeciwnika. W tym wypadku jest 6 par kart które może posiadać przeciwnik, więc dla każdej z nich, zgodnie ze strategią zachłanną, należy obliczyć funkcję kryterialną oraz szansę na przegraną. Dodatkowo należy zauważyć, że z perspektywy przeciwnika w każdym przypadku gracz pierwszy ma inny rozkład prawdopodobieństwa wystąpienia karty. Pełne obliczenia znajdują się w tabeli poniżej.

\clearpage
\begin{center}
	Gracz pierwszy po zagraniu Księcia posiada tylko kartę Pokojówki. $F(z_1)$ oznacza ocenę zagrania pierwszej karty, $F(z_2)$ oznacza ocenę zagrania drugiej karty.
\end{center}
\begin{table}[h]
	\caption{Scenariusze reakcji przeciwnika na decyzję w zadanym przypadku}
	\centering
	\begin{tabular}{|l|c|r|r|r|}
		\hline
		\bf{Karty 1 i 2} & $P[]$ pierwszego gracza  & $F(z_1)$ & $F(z_2)$ & Szanse przegranej	\\ \hline
		Str ; Bar & $P[\textit{Pok}] = P[\textit{K-e}] = P[\textit{K-a}] = 33.(3)\%$ & 1.41 & 0.6	& 33.(3)\% \\ \hline
		Str ; K-e & $P[\textit{Pok}] = P[\textit{Bar}] = P[\textit{K-a}] = 33.(3)\%$ & 1.41 & 1.37 & 33.(3)\%	\\ \hline
		Str ; K-a & $P[\textit{Str}] = P[\textit{Bar}] = P[\textit{K-e}] = 33.(3)\%$ & 1.41 & 0 & 33.(3)\% \\ \hline
		Bar ; K-e & $P[\textit{Str}] = P[\textit{Pok}] = P[\textit{K-a}] = 33.(3)\%$ & 1.39 & 1.37 & 100\% \\ \hline
		Bar ; K-a & $P[\textit{Str}] = P[\textit{Pok}] = P[\textit{K-e}] = 33.(3)\%$ & 2.06 & 0 & 100\% \\ \hline
		K-e ; K-a & $P[\textit{Str}] = P[\textit{Pok}] = P[\textit{Bar}] = 33.(3)\%$ & 1.04 & 0 & 0\% \\ \hline
	\end{tabular}
\end{table}
Dodatkowo, wprowadźmy funkcję $K(z) = F_{max} - F_{min}$, która wyceni dane zagranie.

Każdy rozważany scenariusz ma taką samą szansę wystąpienia, czyli $\frac{1}{6}$. Statystycznie szansa na przegraną wynosi więc:
\begin{center}
 $\frac{1}{6} * (\frac{1}{3} + \frac{1}{3} + \frac{1}{3} + 1 + 1 + 0) = \frac{1}{6} * 3 = 0.50 $
\end{center}
Warunkiem wystąpienia możliwości przegrania, jest brak wygranej po zagraniu Księcia na przeciwnika, czyli ostatecznie szanse na przegraną wynoszą:
\begin{center}
	$F_{min}(\textit{K-e\_P}, P[]) = 0.75 * 0.50 = 0.375$
\end{center}
Modyfikując o tę wartość ocenę zagrania Księcia otrzymujemy finalnie:
\begin{center}
	$K(\textit{K-e\_P}) =  F_{max}(\textit{K-e\_P}, P[]) - F_{min}(\textit{K-e\_P}, P[]) = 1.29 - 0.375 = 0.915$
\end{center} 
W przypadku zagrania Pokojówki szanse przegrania wynoszą 0\%, ponieważ zgodnie z zasadami gry jesteśmy odporni na działanie kart przeciwnika:
\begin{center}
	$K(Pok\_Z) =  F_{max}(Pok\_Z, P[]) - F_{min}(Pok\_Z, P[]) = 1.05 - 0 = 1.05$
\end{center} 
Wynika z tego, że decyzją, która minimalizuje maksymalną stratę jest zagranie $z = Pok\_Z$.

Na podstawie powyższych rozważań zapisałem implementację algorytmu zachłannego w postaci pseudokodu.
\subsection{Zapis pseudokodem}
Wykorzystane zmienne pomocnicze:
\begin{itemize}
	\item $K(z)$ - oznacza wspomnianą wyżej funkcję wyceny zagrania. 
	\item $L[Y_n]$ - zbiór możliwych zbiorów decyzji przeciwnika na danym poziomie drzewa.	
	\item $Y_n$ - oznacza możliwy zbiór zagrań dostępnych dla przeciwnika na danym poziomie drzewa.
	\item Funkcja $R(L[Y_n])$ oznacza prawdopodobieństwo przegranej po reakcji przeciwnika.
	\item $RK$ - zbiór zagrań będących reakcją przeciwnika.
	\item $P_{porazki}[]$ - tablica zawierająca prawdopodobieństwo porażki po danym zagraniu
\end{itemize}

\begin{algorithmic}[1]
	\Function{$D_{minimaksowa}$}{$I_n, X_n$}
	\State $P[] \gets$ obliczPrawdopodobienstwo()
		\ForAll{ $z_i \in X_n$ } \Comment $i=1..|X_n|$
			\State $K(z_i) \gets F_{max}(z_i, P[]) - F_{min}(z_i, P[])$
		\EndFor		
		\State $ decyzja \gets z_0$ \Comment Szukanie zagrania o najwyższej wycenie
		\ForAll{ $z_i \in X_n $ } 
			\If {$K(decyzja) < K(z_i)$}
				\State $decyzja \gets z_i$
			\EndIf
		\EndFor		
		\State \textbf{return} $decyzja$
	\EndFunction
\end{algorithmic}

Funkcja $F_{max}(z, P[])$ jest identyczna do $F(z, P[])$ występującej w strategii zachłannej, w związku z czym przedstawiam wyłącznie $F_{min}(z, P[]$).
\begin{algorithmic}[1]
	\Function{$F_{min}$}{$z, P[]$}	
		\State $L[Y_n] \gets$ utwórz listę możliwych zbiorów decyzji u przeciwnika 
		\Switch{$z$}
			\Case{Str\_Kap} \Comment Szanse, że Strażniczka \textbf{nie} przyniesie zwycięstwa
				\State \textbf{return} $ (1 - P[Kap]) * R(L[Y_n]) $
			\EndCase
			\Case{Str\_Bar}
				\State \textbf{return} $ (1 - P[Bar]) * R(L[Y_n]) $
			\EndCase
				\State ...
			\Case{Str\_K-a}
				\State \textbf{return} $ (1 - P[\textit{K-a}]) * R(L[Y_n]) $
			\EndCase
			\Case{Kap\_Z}
				\State \textbf{return} $  R(L[Y_n]) $
			\EndCase
			\Case{Bar\_Z}	\Comment Szanse, że porównanie zakończy się remisem 
				\State $ szanseRemisu \gets P[DK]$ 
				\State \textbf{return} $ szanseRemisu * R(L[Y_n]) $
			\EndCase
			\Case{Pok\_Z}
				\State \textbf{return} $ 0 $
			\EndCase
			\Case{K-e\_P}
				\State \textbf{return} $ (1 - P[\textit{K-a}]) * R(L[Y_n]) $
			\EndCase
			\Case{K-e\_S} 
				\If {$ DK == \textit{K-a} $}  
					\State \textbf{return} $ 0 $
				\Else
					\State \textbf{return} $ R(L[Y_n]) $
				\EndIf
			\EndCase
			\Case{Krl\_Z}
				\State \textbf{return} $ R(L[Y_n]) $
			\EndCase
			\Case{Hra\_K-e\_Krl}
				\State \textbf{return} $ R(L[Y_n]) $
			\EndCase
			\Case{Hra\_Z}
				\State \textbf{return} $ R(L[Y_n]) $
			\EndCase
				\Case{K-a}
			\State \textbf{return} $ 10 $
			\EndCase
		\EndSwitch
	\EndFunction
\end{algorithmic}

Funkcja $R(L[Y_n])$ oznacza szansę porażki po reakcji przeciwnika.
\begin{algorithmic}[1]
	\Function{$R$}{$L[Y_n]$}
		\ForAll {$Y_n \in L[Y_n]$}
			\State $P[] \gets$ obliczPrawdopodobienstwo()
			\State $RK \gets RK \cup D_{zachlanna}(Y_n, P[]) $	\Comment RK to zbiór reakcji dla każdego ze zbiorów $Y_n$
		\EndFor
		\ForAll {$z_i \in RK$}	\Comment Sprawdzenie szansy porażki, $i=1..|RK|$
			\Switch{$z_i$}
				\Case{Str\_DK}
					\State $P_{porazki}[z_i] \gets 1$
				\EndCase
				\Case{Bar\_Z}
					\If{$W(DK) < W(DK_{przeciwnika})$}
						\State $P_{porazki}[z_i] \gets 1$
					\Else
						\State $P_{porazki}[z_i] \gets 0$
					\EndIf
				\EndCase
				\Case{K-e\_Z}
					\If{$DK == \textit{K-a}$}
						\State $P_{porazki}[z_i] \gets 1$
					\Else
						\State $P_{porazki}[z_i] \gets 0$
					\EndIf
				\EndCase
				\Case{default}
					$P_{porazki}[z_i] \gets 0$
				\EndCase
			\EndSwitch
		\EndFor
		\State \textbf{return} $ \sum_{i=0}^{|RK|} \frac{1}{|RK|} * P_{porazki}[i] $
	\EndFunction
\end{algorithmic}


\section{Algorytm Monte Carlo Tree Search}
\label{sec:mcts}

\subsection{Opis}

Drzewo Przeszukiwań Monte Carlo jest metodą heurystycznego podejmowania decyzji, często wykorzystywaną w grach typu Hex$^{[\ref{bib:mcts_hex}]}$ czy Go$^{[\ref{bib:wiki_mcts}]}$. W przeciwieństwie do deterministycznych algorytmów przeszukujących drzewo decyzyjne, MCTS buduje drzewo wariantów poprzez losowe próbkowanie najbardziej obiecujących ruchów. Dzięki temu jego dużą zaletą jest możliwość efektywnego wykorzystania w grach o wysokim rozgałęzieniu drzewa decyzyjnego$^{[\ref{bib:mcts_wprowadzenie}]}$. Popularność wykorzystania tego algorytmu stale rośnie, czego przykładem jest zastosowanie w niedeterministycznej grze planszowej Osadnicy z Catanu$^{[\ref{bib:mcts_osadnicy}]}$, czy w strategii turowej Total War: Rome II$^{[\ref{bib:mcts_totalWar}]}$. Główne cechy algorytmu:
\begin{itemize}
	\item przerywalność algorytmu w dowolnym momencie
	\item zależność wydajności od czasu - im dłużej działa algorytm, tym lepsze osiąga wyniki
	\item uniwersalność - algorytm do działania potrzebuje wyłącznie zbioru dostępnych ruchów oraz możliwości przeprowadzenia losowej rozgrywki od danego stanu
	\item niezależność od wiedzy eksperckiej występującej w grze $^{[\ref{bib:mcts_wprowadzenie}]}$
\end{itemize}

\subsection{Sposób wykorzystania}

Jak opisali autorzy artykułu \textit{Monte-Carlo Tree Search: A New Framework for Game AI}$^{[\ref{bib:mcts_opis}]}$, zasady działania algorytmu są następujące: otrzymując zbiór informacyjny $I_n$ oraz zbiór dopuszczalnych ruchów (decyzji) $X_n$, tworzony jest korzeń $T$, oraz węzły $s_0$ .. $s_i$ ($i=1..|X_n|$) odpowiadające stanom gry po wykonaniu dostępnych ruchów (decyzji). Następnie przez zadany czas $t$ powtarzane są następujące kroki:
\begin{enumerate}
	\item \textbf{Selekcja} - zaczynając od korzenia $T$, wybieraj kolejne węzły balansując pomiędzy \textbf{eksploatacją} i \textbf{eksploracją}, aż dotrzesz do węzła-liścia $L$.
	\item \textbf{Ekspansja} - jeśli wybór L nie kończy gry, utwórz węzły potomne i wśród nich wylosuj węzeł $C$
	\item \textbf{Symulacja} - dla wybranego węzła przeprowadzać losową symulację, aż do osiągnięcia wyniku.
	\item \textbf{Propagacja wsteczna} - każdy węzeł od $C$ do $L$ jest aktualizowany o wartość wyniku.
\end{enumerate}
\begin{figure}[h]
	\centering
	\includegraphics[width=\textwidth]{Resources/mcts.png}
	\caption{Schemat działania algorytmu MCTS$^{[\ref{bib:mcts_opis}]}$} 
	\label{fig:llMainImage}
\end{figure}
Szczególnym elementem jest selekcja, ponieważ ona odpowiada za kierunek rozrastania się drzewa. Eksploatacja oznacza wybieranie ruchów o wysokiej częstości wygranych, a eksploracja to badanie ruchów o niskiej liczbie przeprowadzonych symulacji. Brak równowagi pomiędzy tymi dwoma strategiami próbkowania doprowadzi do zakłamań, na przykład poprzez pułapkę optimum lokalnego. W związku z tym w 2006 roku Kocsis i Szepervari opracowali wzór równoważący selekcję nazwany Upper Confidence bound applied to Trees(UCT) (górna granica ufności)$^{[\ref{bib:mcts_uct}]}$:
\begin{center}
	${v_i} + E*\sqrt{\frac{\ln{N}}{n_i}}$
\end{center}
\begin{itemize}
	\item $v_i$ - estymowana wartość węzła
	\item $E$ - empirycznie dobierany parametr eksploracji, zazwyczaj $\sqrt{2}$
	\item $N$ - ilość odwiedzin węzła-rodzica
	\item $n_i$ - ilość odwiedzin w danym węźle
\end{itemize}
Ważne jest, że każdy węzeł posiada informację o szacunkowej wartości opartej na wynikach symulacji, oraz o liczbie przeprowadzonych symulacji.

Ze względu na fakt, że algorytm do działania wymaga wyłącznie listy dostępnych ruchów oraz możliwości przeprowadzenia symulacji (na przykład za pomocą interfejsu w grze), jego implementacja do gry ,,Love Letter'' nie wymaga specjalnych modyfikacji. Powyższa wiedza w pełni pozwala na wykorzystanie algorytmu w grze.
\subsection{Zapis pseudokodem}
Wykorzystane zmienne pomocnicze:
\begin{itemize}
	\item $Nast(s, z)$ - funkcja wskazująca następnik węzła $u$ po wyborze zagrania (łuku) $z$.
	\item $X_L$ - zbiór możliwych zagrań dostępnych w węźle $L$.
	\item $wynik$ - wypłata gracza po symulacji.
\end{itemize}
\begin{algorithmic}[1]
	\Function{$d_{mcts}$}{$I_n, X_n$}
		\State $T \gets $ utwórz korzeń T z obecnego zbioru informacyjnego $I_n$
		\Repeat
			\State $L \gets $ selekcja($T$)
			\If {$L$ nie jest końcem gry}
				\ForAll{ $z_i \in X_L$ } \Comment $i=1..|X_L|$, 
					\State $L \gets$ dodajDziecko($Nast(L, z_i)$)
				\EndFor
				\State $C \gets$ wylosujDziecko($L$)
				\State $wynik \gets$ symuluj($C$)	\Comment Symulacja
				\Repeat	\Comment Propagacja wsteczna
					\State $C \gets $ aktualizuj($wynik$)
					\State $C \gets $ rodzic($C$)
				\Until {$C==T$}
			\EndIf 
		\Until{koniec czasu}
		\State $decyzja \gets$ najwięcejSymulacji($R$) \Comment ruch z najbardziej obiecującego węzła 
		\State \textbf{return} $decyzja$
	\EndFunction
\end{algorithmic}