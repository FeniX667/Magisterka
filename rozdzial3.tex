\chapter{Przegląd wybranych algorytmów i innych rozwiązań}
\label{cha:rozdz3}

W tym rozdziale przedstawiam kilka wybranych algorytmów, które zaimplementuję w następnym rozdziale. Dla każdego z nich opisuję jego użycie w odniesieniu do gry 'Love Letter'.

\section{Algorytm losowy}
\label{sec:algLos}
\subsection{Opis}
Jest to najprostszy algorytm podejmowania decyzji. Na wejściu otrzymuje listę dostępnych zagrań, z których w całkowicie losowy sposób wybiera jedno. Każde z dostępnych zagrań ma takie samo prawdopodobieństwo bycia wybranym przez ten algorytm.

\subsection{Sposób wykorzystania}
Z uwagi na prostotę tego algorytmu, używam go jako punktu odniesienia dla wszystkich pozostałych strategii. Dla każdej z nich algorytm losowy spełnia rolę drugiego gracza i w ten sposób można łatwo porównać je ze sobą. Dla tego algorytmu zastosowałem jedną drobną modyfikację: nigdy nie podejmie decyzji o zagraniu Księżniczki - oznacza to natychmiastową przegraną niezależnie od momentu gry, co mogłoby mocno wypaczyć wyniki innych algorytmów.

\subsection{Zapis pseudokodem}
\begin{algorithmic}[1]
	\Function{$D_{losowa}$}{$X_n,s_{n-1}$}	
		\ForAll{ $x_i$ } 
			\If {$x_i ==$ Księżniczka}
				\State $X_n \gets X_n - x_i$
			\EndIf
		\EndFor
	\State \textbf{return} wylosujJeden($X_n$) 
	\EndFunction
\end{algorithmic}

\section{Algorytm zachłanny}
\label{sec:algZach}
\subsection{Opis}
Algorytm zachłanny polega na wyborze najlepszego możliwego zagrania dostępnego w danej chwili, nie analizując jego konsekwencji w przyszłości. Pomimo, że takie podejmowanie decyzji krótkowzroczne, jest łatwe w implementacji i daje atrakcyjne wyniki w niektórych problemach, np. przy szukaniu minimalnego drzewa rozpinającego\textsuperscript{[\ref{bib:algorytmy_zachlanny}]}.

\subsection{Sposób wykorzystania}

W kontekście gry 'Love Letter', implementacja algorytmu zachłannego wymaga pewnego doprecyzowania. Najważniejszą częścią jest funkcja kryterialna oceniająca wartość zagrania, która w niektórych przypadkach musi być oparta o probabilistykę wystąpienia kart u przeciwnika. 

Rozważmy przypadek, w którym algorytm musi podjąć decyzję o zagraniu karty Strażniczki, lub karty Barona. Jest to pierwszy ruch gracza, a w widocznych kartach odrzuconych na starcie są odrzucone karty Króla, Księcia i Pokojówki. Oznacza to, że w talii pozostało 9 kart, a jedna z odrzuconych jest niewidoczna, niemniej jednak ją też trzeba brać pod uwagę. Wyliczenie prawdopodobieństwa \textit{P} jaką kartę ma przeciwnik jest tym momencie proste, jednak trzeba jeszcze wziąć pod uwagę drobny szczegół - czy do liczenia \textit{P} wliczać karty posiadane w ręce. Z jednej strony wydaje się to nielogiczne i może prowadzić do wybierania nieoptymalnych decyzji (co przeczyłoby idei algorytmu zachłannego), z drugiej strony można to potraktować jako element blefu, który jest nieodłączną częścią każdej gry towarzyskiej. W swojej implementacji założyłem absolutną zachłanność algorytmu i karty posiadane na ręce są pomijane w obliczeniach. 
Wobec powyższego, prawdopodobieństwo wystąpienia kart u przeciwnika rozkłada się następująco:

\begin{table}[h]
	\caption{Przykład rozkładu prawdopodobieństwa}
	\centering
		\begin{tabular}{|l|r|}
			\hline
			\bf{Karta} & $P(Karta)$	\\ \hline
			Strażniczka & 30\% 			\\ \hline
			Kapłan & 20\% 				\\ \hline
			Baron & 10\% 				\\ \hline
			Pokojówka & 10\% 			\\ \hline
			Książę & 10\% 				\\ \hline
			Hrabina & 10\% 				\\ \hline
			Księżniczka & 10\% 			\\ \hline
		\end{tabular}
\end{table}

Zagranie karty Baron oznacza porównanie drugiej karty z kartą przeciwnika. Łatwo policzyć, że w 70\% przypadków skończyłoby to się porażką, a w 30\% nie było by żadnego efektu. Drugim dostępnym ruchem jest zagranie karty Strażniczki, a w jej przypadku najlepszym wyborem jest wytypowanie Kapłana, co daje 20\% szans na zwycięstwo i 80\% szans, że nie nastąpi żaden efekt. Zauważmy, że gdybyśmy wliczali posiadane karty do obliczenia prawdopodobieństwa, wystąpienie Barona i Kapłana byłoby tak samo możliwe. W takich przypadkach algorytm powinien zawsze celować w kartę z wyższym numerem. By formalnie stwierdzić, jaka decyzja powinna zostać podjęta, musimy obliczyć funkcję kryterialną dla dostępnych zagrań i wybrać to zagranie, dla której funkcja przyjmuje wyższy wynik. Przyjmijmy, że funkcja kryterialna wygląda następująco:

\begin{center}
	$F(Karta) = 1 + prawdopodobienstwo\_wygranej - prawdopodobienstwo\_przegranej$
\end{center}
Po podstawieniu otrzymujemy:
\begin{center}
 $F(Strazniczka)=1.2$ i $F(Baron) = 0.3$
 
 $F(Strazniczka)>F(Baron) => Decyzja=Zagranie(Pokojowka + Kaplan)$ 
\end{center} 

Najlepszą decyzją w tym wypadku jest zagranie karty Strażniczki z wyborem karty Kapłana. Jak jednak na podstawie powyższego wzoru ocenić zagranie karty Kapłana, Pokojówki lub Króla? Każda z nich wymaga unikalnej funkcji kryterialnej. Mając na uwadze, że w kontekście strategii zachłannej decyzja zawsze powinna być optymalna w ujęciu chwili, ustaliłem następujące warunki oceny zagrania karty:
\begin{itemize}
	\item Strażniczka - ocena zagrania wzrasta gdy pozwala wyeliminować przeciwnika.
	\item Kapłan - w ujęciu chwili jest to karta neutralna i ocena jej zagrania będzie zawsze stała.
	\item Baron - ocena zagrania wzrasta gdy mamy drugą kartę silniejszą niż może mieć przeciwnik.
	\item Pokojówka - podobnie jak w przypadku kapłana, ocena zagrania karty będzie stała.
	\item Książę - ocena zagrania na przeciwnika wzrasta z prawdopodobieństwem Księżniczki u przeciwnika. Ocena zagrania na siebie jest stała, lecz 0 gdy druga karta to Księżniczka.
	\item Król - ocena stała
	\item Hrabina - ocena stała, jednak musimy ją zagrać gdy druga posiadana karta to Król lub Książe
	\item Księżniczka - nie może być nigdy wyrzucona.
	\item Dodatkowo, jeśli oba zagrania mają taką samą wartość, powinna być podjęta decyzja o zagraniu karty o niższej wartości.
\end{itemize}

\subsection{Zapis pseudokodem}
\begin{algorithmic}[1]
	\Function{$D_{zachlanna}$}{$X_n,s_{n-1}$}
		\ForAll{ $rodzajKarty \in rodzajeKart$ } \Comment Tworzenie tablicy prawdopodobieństwa
			\State $P[rodzajKarty] \gets$  prawdopodobieństwo wystąpienia karty danego rodzaju u przeciwnika	
		\EndFor
		\State $ decyzja \gets NULL$ \Comment Szukanie zagrania o najwyższej wycenie
		\ForAll{ $x_i \in X_n$ } \Comment Obliczenie funkcji kryterialnej dla każdego zagrania
				\If {$F(decyzja, P[] < F(x_i, P[]$}
					\State $decyzja \gets x_i$
				\EndIf
		\EndFor		
		\State \textbf{return} $decyzja$
	\EndFunction
\end{algorithmic}

Gdzie funkcja kryterialna wygląda następująco:
\begin{algorithmic}[1]
	\Function{$F$}{$x_i,P[]$}	
		\Switch{$x_i$}
			\Case{Strażniczka + Kaplan} \Comment Zagranie Strażniczki na dany typ karty
				\State \textbf{return} $ 1 + P[Kaplan] + 0.08 $
			\EndCase
			\Case{Strażniczka + Baron}
				\State \textbf{return} $ 1 + P[Baron]  + 0.08 $
			\EndCase
			\State ...
			\Case{Strażniczka + Księżniczka}
				\State \textbf{return} $ 1 + P[Ksiezniczka]  + 0.08 $
			\EndCase
			\Case{Kapłan}
				\State \textbf{return} $ 1 + 0.07 $
			\EndCase
			\Case{Baron}	\Comment Kryterium zależy od wartości W() drugiej posiadanej karty (DK)
				\State $ szansePrzegranej \gets 0$ 
				\State $ szanseWygranej \gets 0$ 
				\ForAll {$typKarty$}
					\If {$ W(DK) < W(typKarty) $}  
						\State $szansePrzegranej \gets szansePrzegranej + P[typKarty]$ 
					\ElsIf {$ W(DK) > W(typKarty) $}
						\State $szanseWygranej \gets szanseWygranej + P[typKarty]$ 
					\EndIf
				\EndFor
				\State \textbf{return} $ 1 - szansePrzegranej + szanseWygranej + 0.06 $
			\EndCase
			\Case{Pokojówka}
				\State \textbf{return} $ 1 + 0.05 $
			\EndCase
			\Case{Książe na przeciwnika}
				\State \textbf{return} $ 1 + P[Ksiezniczka] + 0.04 $
			\EndCase
			\Case{Książe na siebie} 
				\If {$ DK == Ksiezniczka $}  
					\State \textbf{return} $ 0 $
				\Else
					\State \textbf{return} $ 1 + 0.04 $
				\EndIf
			\EndCase
			\Case{Król}
				\State \textbf{return} $ 1 + P[Ksiezniczka] + P[Hrabina] + 0.03 $
			\EndCase
			\Case{Hrabina, gdy druga posiadana karta to Król lub Książe}
				\State \textbf{return} $ 10 + 0.02 $
			\EndCase
			\Case{Hrabina}
				\State \textbf{return} $ 1 + 0.02 $
			\EndCase
			\Case{Księżniczka}
				\State \textbf{return} $ 0 $
			\EndCase
		\EndSwitch
	\EndFunction
\end{algorithmic}


\section{Algorytm min-max}
\label{sec:minmax}
\subsection{Opis}
Algorytm minimaksowy polega na ,,minimalizowaniu maksymalnych możliwych strat'' bądź alternatywnie na ,,maksymalizacji minimalnego zysku''$^{[\ref{bib:wiki_minMax}]}$. Zgodnie z [\ref{bib:wazniak_minMax}] często stosuje się go do gier o następujących zasadach:
\begin{itemize}
	\item występuje dwóch graczy
	\item ruchy wykonywane są naprzemiennie
	\item w każdym stanie istnieje skończona liczba decyzji do podjęcia
	\item stan i podjęta decyzja jednoznacznie wyznaczają stan następny
	\item każdy stan może zakwalifikować do jednej z następujących kategorii:
	\begin{itemize}
		\item wygrana pierwszego gracza
		\item wygrana drugiego gracza
		\item remis
		\item sytuacja nierozstrzygnięta
	\end{itemize}
\end{itemize}

\subsection{Sposób wykorzystania}
W podanym powyżej opisie, gra Love Letter niezgodna jest w punkcie czwartym. Ze względu na niepewność jaką kartę posiada przeciwnik, jaką pociągnie ze stosu w swojej turze, oraz jaki wykona ruch, podjęcie decyzji $x_i$ w stanie $s_{n-1}$ nie określa jednoznacznie stanu $s_n$. Mimo to, jesteśmy w stanie statystycznie ocenić możliwości przeciwnika, a tym samym zminimalizować naszą maksymalną stratę. Ponieważ zysk należy rozumieć jako zwiększenie szansy na wygraną (tak jak w strategii zachłannej), to strata oznacza zwiększenie szansy na przegraną. 

Rozważmy scenariusz, w którym gracz pierwszy posiada kartę Księcia oraz Pokojówki. Na stosie pozostało 2 karty, 1 karta jest zakryta zgodnie z zasadami rozpoczęcia rundy i przeciwnik również posiada 1 kartę. Wśród tych 4 nieznanych graczowi kart są karty Strażniczki, Barona, Księcia oraz Księżniczki. Prawdopodobieństwo wystąpienia każdej z nich u przeciwnika wynosi 25\% i razem stanowią one tablicę prawdopodobieństwa $P[]$. Przyjmijmy, że wykorzystywana w poprzedniej strategii funkcja kryterialna $F(Karta, P[])$ to maksymalizacja zysku i nazwijmy ją $F_{max}(Karta, P[])$  Zgodnie z jej definicją $F_{max}(Ksiaze, P[]) = 1.29$ i $F_{max}(Pokojowka, P[]) = 1.05$, więc zagranie Księcia maksymalizuje szansę na wygraną. Jeśli jednak weźmiemy pod uwagę, że w 75\% przypadków nie wygrywamy, wówczas musimy rozważyć odpowiedź przeciwnika. W tym wypadku jest 6 par kart które może posiadać przeciwnik, więc dla każdej z nich, zgodnie ze strategią zachłanną, należy obliczyć funkcję kryterialną oraz szansę na przegraną. Dodatkowo należy zauważyć, że z perspektywy przeciwnika w każdym przypadku gracz pierwszy ma inny rozkład prawdopodobieństwa wystąpienia karty. Pełne obliczenia znajdują się w tabeli poniżej.
\begin{center}
	Skróty : \textit{Str} - Strażniczka; \textit{Bar} - Baron; \textit{Pok}-Pokojówka; \textit{K-e} - Książę; \textit{K-a} - Księżniczka
	
	
	Gracz pierwszy po zagraniu Księcia posiada kartę Pokojówki.
\end{center}
\begin{table}[h]
	\caption{Scenariusze reakcji przeciwnika na decyzję w stanie $s_{n-1}$}
	\centering
	\begin{tabular}{|l|c|r|r|r|}
		\hline
		\bf{Karty 1 i 2} & $P(Karta)$ pierwszego gracza  & $F(Karta_1)$ & $F(Karta_2)$ & Szanse przegranej	\\ \hline
		Str ; Bar & $P(\textit{Pok}) = P(\textit{K-e}) = P(\textit{K-a}) = 33.(3)\%$ & 1.41 & 0.6	& 33.(3)\% \\ \hline
		Str ; K-e & $P(\textit{Pok}) = P(\textit{Bar}) = P(\textit{K-a}) = 33.(3)\%$ & 1.41 & 1.37 & 33.(3)\%	\\ \hline
		Str ; K-a & $P(\textit{Str}) = P(\textit{Bar}) = P(\textit{K-e}) = 33.(3)\%$ & 1.41 & 0 & 33.(3)\% \\ \hline
		Bar ; K-e & $P(\textit{Str}) = P(\textit{Pok}) = P(\textit{K-a}) = 33.(3)\%$ & 1.39 & 1.37 & 100\% \\ \hline
		Bar ; K-a & $P(\textit{Str}) = P(\textit{Pok}) = P(\textit{K-e}) = 33.(3)\%$ & 2.06 & 0 & 100\% \\ \hline
		K-e ; K-a & $P(\textit{Str}) = P(\textit{Pok}) = P(\textit{Bar}) = 33.(3)\%$ & 1.04 & 0 & 0\% \\ \hline
	\end{tabular}
\end{table}
Każdy rozważany scenariusz ma taką samą szansę wystąpienia, czyli $\frac{1}{6}$. Statystycznie szansa na przegraną wynosi więc:
\begin{center}
 $\frac{1}{6} * (\frac{1}{3} + \frac{1}{3} + \frac{1}{3} + 1 + 1 + 0) = \frac{1}{6} * 3 = 0.50 $
\end{center}
Warunkiem wystąpienia możliwości przegrania, jest brak wygranej po zagraniu Księcia, czyli ostatecznie szanse na przegraną wynoszą:
\begin{center}
	$F_{min}(Ksiaze, P[]) = 0.75 * 0.50 = 0.375$
\end{center}
Modyfikując o tę wartość ocenę zagrania Księcia otrzymujemy finalnie:
\begin{center}
	$K(Ksiaze) =  F_{max}(Ksiaze, P[]) - F_{min}(Ksiaze, s_{n-1}) = 1.29 - 0.375 = 0.915$
\end{center} 
W przypadku zagrania Pokojówki szanse przegrania wynoszą 0\%, ponieważ zgodnie z zasadami gry jesteśmy odporni na działanie kart przeciwnika:
\begin{center}
	$K(Pokojowka) =  F_{max}(Pokojowka, P[]) - F_{min}(Pokojowka, s_{n-1}) = 1.05 - 0 = 1.05$
\end{center} 
Wynika z tego, że decyzją która minimalizuje maksymalną stratę jest $Zagranie(Pokojowka)$.

\subsection{Zapis pseudokodem}
\begin{algorithmic}[1]
	\Function{$D_{minimaksowa}$}{$X_n,s_{n-1}$}
		\ForAll{ $rodzajKarty \in rodzajeKart$ } \Comment Tworzenie tablicy prawdopodobieństwa
			\State $P[rodzajKarty] \gets$  prawdopodobieństwo wystąpienia karty danego rodzaju u przeciwnika	
		\EndFor
		\ForAll{ $x_i \in X_n$ } \Comment Obliczenie funkcji kryterialnej dla każdego zagrania
			\State $K[x_i] \gets F_{max}(x_i, P[]) - F_{min}(x_i, s_{n-1})$
		\EndFor		
		\State $ decyzja \gets x_0$ \Comment Szukanie zagrania o najwyższej wycenie
		\ForAll{ $x_i$ } 
			\If {$K[decyzja] < K[x_i]$}
				\State $decyzja \gets x_i$
			\EndIf
		\EndFor		
		\State \textbf{return} $decyzja$
	\EndFunction
\end{algorithmic}

Funkcja $F_{max}(Karta)$ jest identyczna do $F(Karta)$ występującej w strategii zachłannej, w związku z czym zapiszę wyłącznie $F_{min}(Karta$).
\begin{algorithmic}[1]
	\Function{$F_{min}$}{$x_i, s_{n-1}$}	
		\State $L[Y_n] \gets$ lista możliwych zbiorów decyzji przeciwnika 
		\Switch{$x_i$}
			\Case{Strażniczka + Kaplan} \Comment Szanse, że Strażniczka nie przyniesie zwycięstwa
				\State \textbf{return} $ (1 - P[Kaplan]) * R(s_{n-1}, L[Y_n]) $
			\EndCase
			\Case{Strażniczka + Baron}
				\State \textbf{return} $ (1 - P[Baron]) * R(s_{n-1}, L[Y_n]) $
			\EndCase
				\State ...
			\Case{Strażniczka + Księżniczka}
				\State \textbf{return} $ (1 - P[Ksiezniczka]) * R(s_{n-1}, L[Y_n]) $
			\EndCase
			\Case{Kapłan}
				\State \textbf{return} $  R(s_{n-1}, L[Y_n]) $
			\EndCase
			\Case{Baron}	\Comment Szanse, że porównanie zakończy się remisem 
				\State $ szanseRemisu \gets P[DK]$ 
				\State \textbf{return} $ szanseRemisu * R(s_{n-1}, L[Y_n]) $
			\EndCase
			\Case{Pokojówka}
				\State \textbf{return} $ 0 $
			\EndCase
			\Case{Książe na przeciwnika}
				\State \textbf{return} $ (1 - P[Ksiezniczka]) * R(s_{n-1}, L[Y_n]) $
			\EndCase
			\Case{Książe na siebie} 
				\If {$ DK == Ksiezniczka $}  
					\State \textbf{return} $ 0 $
				\Else
					\State \textbf{return} $ R(s_{n-1}, L[Y_n]) $
				\EndIf
			\EndCase
			\Case{Król}
				\State \textbf{return} $ R(s_{n-1}, L[Y_n]) $
			\EndCase
			\Case{Hrabina, gdy druga posiadana karta to Król lub Książe}
				\State \textbf{return} $ R(s_{n-1}, L[Y_n]) $
			\EndCase
			\Case{Hrabina}
				\State \textbf{return} $ R(s_{n-1}, L[Y_n]) $
			\EndCase
				\Case{Księżniczka}
			\State \textbf{return} $ 0 $
			\EndCase
		\EndSwitch
	\EndFunction
\end{algorithmic}

Funkcja $R(s_{n-1}, L[Y_n])$ oznacza szansę porażki po reakcji przeciwnika.
\begin{algorithmic}[1]
	\Function{$R$}{$s_{n-1}, L[Y_n]$}
		\ForAll {$Y_n \in L[X_n]$}
			\State $Z[] \gets D_{zachlanna}(Y_n, s_{n-1}) $	\Comment Lista potencjalnych reakcji
		\EndFor
		\ForAll {$x_i \in Z[]$}	\Comment Sprawdzenie szansy porażki
			\Switch{$x_i$}
				\Case{Strazniczka + DK}	\Comment DK - druga karta pierwszego gracza
					\State $P_{porazki}[x_i] \gets (Z[x_i] - 1.08)$
				\EndCase
				\Case{Baron}
					\If{$W(DK) < W(DK_{przeciwnika})$}
						\State $P_{porazki}[x_i] \gets 1$
					\Else
						\State $P_{porazki}[x_i] \gets 0$
					\EndIf
				\EndCase
				\Case{Książę}
					\If{$DK == Ksiezniczka$}
						\State $P_{porazki}[x_i] \gets 1$
					\Else
						\State $P_{porazki}[x_i] \gets 0$
					\EndIf
				\EndCase
			\EndSwitch
		\EndFor
		\State \textbf{return} $ \sum_{i=0}^{Z[].size} \frac{1}{Z[].size} * P_{porazki}[i] $
	\EndFunction
\end{algorithmic}
\section{Algorytm Monte Carlo Tree Search}
\label{sec:algMCTS}