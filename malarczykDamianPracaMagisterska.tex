\documentclass[11pt]{aghdpl}
% \documentclass[en,11pt]{aghdpl}  % praca w języku angielskim
\usepackage[polish]{babel}
%\usepackage[english]{babel}
\usepackage[utf8]{inputenc}
\usepackage{tikz}

% dodatkowe pakiety
\usepackage{enumerate}
\usepackage{listings}
\usepackage{algorithm}
\usepackage{algpseudocode}
\usepackage{epstopdf}

% Nowe definicje do pseudokodu
\algnewcommand\algorithmicswitch{\textbf{switch}}
\algnewcommand\algorithmiccase{\textbf{case}}

% Nowe środowiska do pseudokodu
\algdef{SE}[SWITCH]{Switch}{EndSwitch}[1]{\algorithmicswitch\ #1\ \algorithmicdo}{\algorithmicend\ \algorithmicswitch}%
\algdef{SE}[CASE]{Case}{EndCase}[1]{\algorithmiccase\ #1}{\algorithmicend\ \algorithmiccase}%
\algtext*{EndSwitch}%
\algtext*{EndCase}%

%ustawienie listingu javy
\definecolor{javared}{rgb}{0.6,0,0} % for strings
\definecolor{javagreen}{rgb}{0.25,0.5,0.35} % comments
\definecolor{javapurple}{rgb}{0.5,0,0.35} % keywords
\definecolor{javadocblue}{rgb}{0.25,0.35,0.75} % javadoc

\lstset{language=Java,
	basicstyle=\ttfamily,
	keywordstyle=\color{javapurple}\bfseries,
	stringstyle=\color{javared},
	commentstyle=\color{javagreen},
	morecomment=[s][\color{javadocblue}]{/**}{*/},
	numbers=left,
	numberstyle=\tiny\color{black},
	numbersep=10pt,
	tabsize=4,
	showspaces=false,
	showstringspaces=false}


\lstloadlanguages{TeX}

\lstset{
  literate={ą}{{\k{a}}}1
           {ć}{{\'c}}1
           {ę}{{\k{e}}}1
           {ó}{{\'o}}1
           {ń}{{\'n}}1
           {ł}{{\l{}}}1
           {ś}{{\'s}}1
           {ź}{{\'z}}1
           {ż}{{\.z}}1
           {Ą}{{\k{A}}}1
           {Ć}{{\'C}}1
           {Ę}{{\k{E}}}1
           {Ó}{{\'O}}1
           {Ń}{{\'N}}1
           {Ł}{{\L{}}}1
           {Ś}{{\'S}}1
           {Ź}{{\'Z}}1
           {Ż}{{\.Z}}1
}

%---------------------------------------------------------------------------

\author{Damian Malarczyk}
\shortauthor{D. Malarczyk}

\titlePL{Analiza i porównanie wybranych algorytmów dla gry karcianej}
\titleEN{Analysis and comparison of selected algorithms for the card game}

\shorttitlePL{Analiza i porównanie wybranych algorytmów dla gry karcianej}
\shorttitleEN{Analysis and comparison of selected algorithms for the card game}

\thesistype{Praca dyplomowa magisterska}
%\thesistype{Engineering Thesis}

\supervisor{dr inż. Edyta Kucharska}
%\supervisor{dr Adam Sędziwy}

\degreeprogramme{Informatyka}
%\degreeprogramme{Computer Science}

\date{2015}

\department{Katedra Informatyki Stosowanej}
%\department{Department of Applied Computer Science}

\faculty{Wydział Elektrotechniki, Automatyki,\protect\\[-1mm] Informatyki i Inżynierii Biomedycznej}
%\faculty{Faculty of Electrical Engineering, Automatics, Computer Science and Biomedical Engineering}

\acknowledgements{Podziękowania}
%\acknowledgements{Serdecznie dziękuję dr Edycie Kucharskiej za możliwość podjęcia pracy magisterskiej.}


\setlength{\cftsecnumwidth}{10mm}

%---------------------------------------------------------------------------
\setcounter{secnumdepth}{4}

\begin{document}

\titlepages
\setcounter{tocdepth}{3}
\tableofcontents
\clearpage

\chapter{Wprowadzenie}
\label{cha:rozdz1}
%---------------------------------------------------------------------------
\section{Przedmiot pracy}
\label{sec:przedmiotPracy}

Od kilku lat coraz większą popularnością cieszą się wszelkiego rodzaju gry planszowe i karciane. Ze względu na ich różnorodną mechanikę, mogą być dobrą podstawą do porównania i analizy działania różnych algorytmów. Gdy grając w jedną z nich, ,,Love Letter'', mój kolega stwierdził, że ,,ta gra to tylko los'', postanowiłem sprawdzić to w praktyce, implementując kilka wybranych algorytmów i sprawdzając ich działanie właśnie na podstawie tej gry.

%---------------------------------------------------------------------------
\section{Cel pracy}
\label{sec:zawartoscPracy}
Celem tej pracy jest implementacja programistyczna gry oraz kilku różnych algorytmów, których działanie można przeanalizować opierając się na statystykach zdobywanych w symulacji. Następnie, spośród wybranych algorytmów, zostanie wybrany algorytm najlepiej spełniający swoje zadanie - czyli wygrywanie gier.

\section{Zawartość pracy}
\label{sec:zawartoscPracy}
Rozdział \ref{cha:rozdz1} jest wprowadzeniem definiującym przedmiot pracy. W rozdziale \ref{cha:rozdz2} opisana została gra karciana ,,Love Letter'' wraz z zasadami. Na ich podstawie dokonuję analizy problemu i definiuję jego model matematyczny. Rozdział \ref{cha:rozdz3} przedstawia proponowane przeze mnie algorytmy rozwiązujące problem. Dla każdego z nich opisany został sposób działania, wykorzystanie w praktyce, oraz mój pomysł na użycie do rozwiązania zadanego problemu. W rozdziale \ref{cha:rozdz4} zebrane są poszczególne etapy tworzenia aplikacji, począwszy od analizy wymagań, poprzez koncepcję wykonania, prezentację diagramów UML oraz listingów, a skończywszy na prezentacji programu. W rozdziale \ref{cha:rozdz5} przedstawione są statystyki związane z przeprowadzonymi symulacjami oraz wnioski z działania algorytmów. Rozdział \ref{cha:rozdz6} stanowi podsumowanie pracy.
\chapter{Gra karciana Love Letter}
\label{cha:rozdz2}

W tym rozdziale opisuję kontekst oraz zasady gry Love Letter. Tłumaczę działanie każdej karty oraz przedstawiam główny cel gry - wygranie określonej ilości rund. Następnie przedstawiam problem i analizuję jego złożoność. Wszystkie załączone zdjęcia oraz instrukcja zaczerpnięte są z [\ref{bib:loveLetterGame}] oraz [\ref{bib:loveLetterWebsite}].

\begin{figure}[h]
	\centering
	\includegraphics{Resources/ll_main_image.png}
	\caption{Love Letter - okładka} 
	\label{fig:llMainImage}
\end{figure}

\section{Opis zasad gry}
\label{sec:opisGry}
W trakcie gry wcielamy się w rolę jednego z adoratorów księżniczki starającego się o zdobycie jej serca. W tym celu przygotowaliśmy list miłosny, który chcemy jej dostarczyć. Niestety, księżniczka pogrążona jest obecnie w żałobie i nie przyjmuje do siebie nikogo obcego, w związku z czym musimy znaleźć inny sposób na przekazanie jej naszego listu. Oprócz księżniczki, na dworze znajdują się inne postacie, z których każda ma mniejszy lub większy dostęp do komnat naszej wybranki i może oddać jej list. Przekazujemy więc naszą przesyłkę swojemu tajnemu posłańcowi, a na koniec gry księżniczka jako pierwszy przeczyta ten list, który został przekazany przez najbardziej zaufaną postać. Serce wybranki zdobywa gracz, który jako pierwszy przekaże w ten sposób od 4 do 7 listów, w zależności od liczby graczy.

\section*{Cel i ustawienie początkowe}
\label{sec:celIUstawieniePoczatkowe}
Love Letter rozgrywa się jako serię rund. Grę wygrywa gracz o następującej ilości wygranych rund:
\begin{itemize}
	\item 7 w grze na 2 graczy,
	\item 5 w grze na 3 graczy,
	\item 4 w grze na 4 graczy.
\end{itemize}
Każda runda dzieli się na tury, w których naprzemiennie jeden z graczy wykonuje ruch. Grę wygrywa ten z nich, który na końcu ostatniej tury posiada kartę o wyższym numerze.

Ustawienie początkowe każdej rundy wygląda następująco:
\begin{itemize}
	\item przetasuj karty
	\item odrzuć 1 wierzchnią kartę nie odkrywając jej (nie bierze udziału w rundzie),
	\item jeśli gra tylko 2 graczy, odrzuć 3 wierzchnie karty, odkryte,
	\item rozdaj po 1 karcie wszystkim graczom,
	\item jeśli jest to pierwsza runda, grę zaczyna gracz, który jako ostatni był na randce, w przeciwnym wypadku zwycięzca poprzedniej rundy.
\end{itemize}

\section*{Tura gracza i opis kart}
\label{sec:turaGracza}
Podczas swojej tury gracz dociąga jedna kartę ze stosu. Następnie wybiera jedną z dwóch kart, które posiada już w ręce, kładzie ją przed sobą tak, by była widoczna dla wszystkich i zastosowuje opisany na niej efekt - nawet jeśli jest negatywny. Zagrana karta pozostaje odkryta przez całą rundę, a druga pozostaje w ręce. Następnie tura przechodzi na osobę po lewej stronie aktywnego gracza.

W grze znajduje się 16 kart, w 8 typach. Każda z typów kart posiada wartość od 1 do 8. Są to kolejno: 4 karty Strażniczki, po 2 karty Kapłana, Barona, Pokojówki i Księcia, oraz po jednej karcie Króla, Hrabiny i Księżniczki. Ich szczegółowy opis wraz z wyglądem znajduje się poniżej:

\clearpage
\begin{figure}[h]
	\centering
	\includegraphics[scale=0.5]{Resources/Love_Letter_Card_Guard.png}
	\caption{Strażniczka} \label{fig:Love_Letter_Card_Guard}
\end{figure}
Na rysunku \ref{fig:Love_Letter_Card_Guard} przedstawiona jest karta typu Strażniczka. Zagrywając tę kartę należy wskazać jednego z pozostałych graczy i odgadnąć kartę którą posiada. Jeśli karta została prawidłowo odgadnięta, wskazany gracz odrzuca ją i przegrywa rundę.

\begin{figure}[h]
	\centering
	\includegraphics[scale=0.5]{Resources/Love_Letter_Card_Priest.png}
	\caption{Kapłan} \label{fig:Love_Letter_Card_Priest}
\end{figure}
Rysunek \ref{fig:Love_Letter_Card_Priest} przedstawia kartę typu Kapłan. Zagrywając tę kartę należy podglądnąć kartę wybranego gracza.

\clearpage
\begin{figure}[h]
	\centering
	\includegraphics[scale=0.5]{Resources/Love_Letter_Card_Baron.png}
	\caption{Baron} \label{fig:Love_Letter_Card_Baron}
\end{figure}
Na rysunku \ref{fig:Love_Letter_Card_Baron} przedstawiona jest karta typu Baron. Po zagraniu tej karty należy w ukryciu porównać drugą posiadaną kartą z wybranym graczem. Następnie ten gracz, który ma kartę o mniejszej wartości odrzuca swoją kartę i przegrywa rundę. W przypadku remisu nic się nie dzieje.

\begin{figure}[h]
	\centering
	\includegraphics{Resources/Love_Letter_Card_Handmaid.png}
	\caption{Pokojówka} \label{fig:Love_Letter_Card_Handmaid}
\end{figure}
Rysunek \ref{fig:Love_Letter_Card_Handmaid} przedstawia kartę typu Pokojówka. Zagranie tej karty sprawia, że gracz jest niewrażliwy na efekt pozostałych kart do czasu swojej następnej tury.

\clearpage
\begin{figure}[h]
	\centering
	\includegraphics[scale=0.5]{Resources/Love_Letter_Card_Prince.png}
	\caption{Książe} \label{fig:Love_Letter_Card_Prince}
\end{figure}
Na rysunku \ref{fig:Love_Letter_Card_Prince} przedstawiona jest karta typu Książę. Zagranie pozwala wybrać dowolnego gracza (w tym siebie), zmusić go do odrzucenia posiadanej karty i pociągnięcia następnej.

\begin{figure}[h]
	\centering
	\includegraphics[scale=0.5]{Resources/Love_Letter_Card_King.png}
	\caption{Król} \label{fig:Love_Letter_Card_King}
\end{figure}
Rysunek \ref{fig:Love_Letter_Card_King} przedstawia kartę typu Król. Po jej zagraniu należy wymienić się pozostałą kartą z innym graczem.

\clearpage
\begin{figure}[h]
	\centering
	\includegraphics{Resources/Love_Letter_Card_Countess.png}
	\caption{Hrabina} \label{fig:Love_Letter_Card_Countess}
\end{figure}
Rysunek \ref{fig:Love_Letter_Card_Countess} przedstawia kartę typu Hrabina. Ta karta ma działanie pasywne. Nie wywiera efektu po zagraniu, natomiast zmusza gracza do jej zagrania jeśli równocześnie posiada na ręce kartę typu Książę lub Król.

\begin{figure}[h]
	\centering
	\includegraphics[scale=0.5]{Resources/Love_Letter_Card_Princess.png}
	\caption{Księżniczka} \label{fig:Love_Letter_Card_Princess}
\end{figure}
Na rysunku \ref{fig:Love_Letter_Card_Princess} przedstawiona jest karta typu Księżniczka. Zagranie tej karty oznacza natychmiastową przegraną w rundzie. Ta zasada działa również, gdy gracz został zmuszony do zagrania tej karty, np. przez efekt karty Książe.
\clearpage

\section{Analiza złożoności problemu}
\label{sec:opisProblemu}
Z wyżej przedstawionych zasad wynika, że gra cechuje się wysokim stopniem losowości i jest niedeterministyczna. W związku z tym powstaje pytanie, czy istnieje strategia mieszana$^{[\ref{bib:wiki_StrategiaTeoriaGier}]}$ optymalizująca podejmowanie decyzji w taki sposób, by zwiększać szansę wygrania gry. Dla uproszczenia założyłem, że gra będzie rozgrywana przez dwóch graczy. Dodatkowo, by ustandaryzować pewne pojęcia, wprowadziłem następujące definicje:
\begin{itemize}
	\item \textit{Zagranie} - jest to typ karty wraz ze sposobem jej wykorzystania. Przykładowo: Zagranie karty typu Strażniczka z wyborem karty typu Król, lub zagranie karty typu Książę z wyborem na gracza przeciwnego.
	\item \textit{Decyzja} - to wybór zagrania, które zostanie użyte jako ruch w grze. Podjęcie decyzji to inaczej zwrócenie zagrania przez algorytm.
	\item \textit{Strategia} - inaczej algorytm, który podejmuje decyzje. 
	\item \textit{Scenariusz} - chronologiczny spis decyzji podjętych przez obu graczy, od początku do końca rundy.
\end{itemize}

By móc opracować najlepszą strategię, musiałbym znać wszystkie możliwe scenariusze gry, na podstawie których można by ustalić statystycznie które zagranie w danym momencie jest najbardziej opłacalne. Na scenariusz wpływ mają następujące czynniki:
\begin{itemize}
	\item Kolejność kart w talii na początku rundy
	\item Sposób zagrania karty
\end{itemize}
Dokonałem więc oszacowania, ile takich scenariuszy istnieje. W każdej rundzie bierze udział wszystkie 16 kart. Zakładając, że każda z nich jest unikalna, to  liczba wszystkich możliwych kolejności kart to permutacja, którą obliczam wzorem podanym w [\ref{bib:tabliceMatematyczne}]:

\begin{center}
	$P_n = n!$ , gdzie $n\in N^+$
\end{center}

Dla  $n$ = 16, $n!=20 922 789 888 000$. Część kart się powtarza, więc tę liczbę należy jeszcze podzielić przez permutacje powtarzających się kart Strażniczki, Kapłana, Barona, Pokojówki oraz Księcia. Razem jest to $4! * 2! * 2! * 2! * 2! =  384$. Ostatecznie wynika, że liczba unikalnych kolejności kart wynosi: 

\begin{center}
	$20 922 789 888 000 / 32 = 54486432000$ - 54mld, 864mln i 432 tys.
\end{center}

Nie jest to jednak liczba wszystkich dostępnych scenariuszy. W każdej turze gracz ma do wyboru co najmniej dwa zagrania, ponieważ tyle ma dostępnych kart. Jednakże, w przypadku karty Strażniczki możliwości jest więcej, ponieważ można wytypować 7 typów kart, a w jednym przypadku może to pokonać przeciwnika i skończyć rundę. W związku z tym, by oszacować liczbę scenariuszy na zadanej kolejności kart, posłużyłem się następującym przybliżeniem:
\begin{itemize}
	\item zgodnie z zasadami gry dla dwóch graczy, odrzucam łącznie 4 pierwsze karty (1 zakryta, 3 odkryte).
	\item rozdaję po 1 karcie obu graczom. Pozostaje 10 kart w talii.
	\item zakładając, że żadna decyzja nie spowoduje przerwania rundy, gracze łącznie 10 razy pociągną kartę, więc podejmą 10 decyzji.
	\item w uproszczeniu, każdą decyzję można przedstawić jako 0 (zagranie posiadanej karty) lub 1 (zagranie pociągniętej karty).
\end{itemize}
Na podstawie powyższego można oszacować, że możliwych scenariuszy dla danej kolejności kart jest $2^{10}=1024$. Łącząc tę liczbę z ilością możliwych kolejności kart, utrzymujemy przybliżoną liczbę scenariuszy:

\begin{center}
	$54486432000 * 1024 = 55794106368000 \approx  5.8*10^{12}$
\end{center}

Z uwagi na rząd wielkości, stworzenie strategii na podstawie analizy statystycznej wszystkich dostępnych scenariuszy jest problemem \textit{NP-trudnym}. Z tego powodu, zamiast odpowiadać na pytanie "Jaka jest najlepsza strategia podejmowania decyzji?", dużo łatwiej będzie odpowiedzieć na pytanie "Która z podanych strategii jest najlepsza?", gdyż jest to charakterystyczna cecha problemów klasy \textit{NP}. Kierując się tą zasadą, w następnym rozdziale opisałem wybrane strategie, których skuteczność sprawdzę implementując je w napisanej przeze mnie aplikacji.


\section{Model matematyczny problemu}
Ponieważ każda runda składa się z ciągu wykonywanych przez graczy tur, których stan zależy od podjętej wcześniej decyzji, to model matematyczny problemu można przedstawić za pomocą programowania dynamicznego:

\begin{figure}[h]
	\centering
	\includegraphics[]{Resources/Schemat_Strategii.png}
	\caption{Schemat programowania dynamicznego} \label{fig:Schemat_Strategii}
\end{figure}
Gdzie wyróżnić należy następujące pojęcia:
\begin{itemize}
	\item n - tura gry
	\item $s_{n-1}$ - stan wejściowy
	\item $x_n \in X_n$ - podjęta decyzja
	\item $s_{n-1} \in S_n$ - stan wyjściowy
	\item $X_n$ - zbiór dopuszczalnych decyzji dla etapu n-tego
	\item $S_n$ - zbiór dopuszczalnych stanów dla etapu n-tego
	\item $T_i$ - funkcja przejścia
	\item $Q$ - funkcja celu
	\item $q_n$ - funkcja oceniająca etap n-ty.
\end{itemize}
Funkcja przejścia $T_i$ wynika bezpośrednio z opisanych wcześniej zasad, pozostaje więc opisać $Q$ oraz $q_n$. Zgodnie z zasadami, by wygrać rundę, musimy posiadać kartę o wyższym numerze niż przeciwnik na końcu ostatniej tury, z tego wynika, że:
\begin{itemize}
	\item $Q = q_k \rightarrow max$, gdzie k to numer ostatniej tury
	\item $q_n = SK_{(karta)}$, gdzie $SK_{(karta)}$ to siła karty posiadanej w ręce na końcu tury
\end{itemize}



\chapter{Przegląd wybranych algorytmów i innych rozwiązań}
\label{cha:rozdz3}

W tym rozdziale przedstawiam kilka wybranych algorytmów, które zaimplementuję w następnym rozdziale. Dla każdego z nich opisuję jego użycie w odniesieniu do gry 'Love Letter'.

\section{Algorytm losowy}
\label{sec:algLos}
\subsection{Opis}
Jest to najprostszy algorytm podejmowania decyzji. Na wejściu otrzymuje listę dostępnych zagrań, z których w całkowicie losowy sposób wybiera jedno. Każde z dostępnych zagrań ma takie samo prawdopodobieństwo bycia wybranym przez ten algorytm.

\subsection{Sposób wykorzystania}
Z uwagi na prostotę tego algorytmu, używam go jako punktu odniesienia dla wszystkich pozostałych strategii. Dla każdej z nich algorytm losowy spełnia rolę drugiego gracza i w ten sposób można łatwo porównać je ze sobą. Dla tego algorytmu zastosowałem jedną drobną modyfikację: nigdy nie podejmie decyzji o zagraniu Księżniczki - oznacza to natychmiastową przegraną niezależnie od momentu gry, co mogłoby mocno wypaczyć wyniki innych algorytmów.

\section{Algorytm zachłanny}
\label{sec:algZach}
\subsection{Opis}
Algorytm zachłanny polega na wyborze najlepszego możliwego zagrania dostępnego w danej chwili, nie analizując jego konsekwencji w przyszłości. Pomimo, że takie podejmowanie decyzji krótkowzroczne, jest łatwe w implementacji i daje atrakcyjne wyniki w niektórych problemach, np. przy szukaniu minimalnego drzewa rozpinającego\textsuperscript{[\ref{bib:algorytmy_zachlanny}]}.

\subsection{Sposób wykorzystania}

W kontekście gry 'Love Letter', implementacja algorytmu zachłannego wymaga pewnego doprecyzowania. Najważniejszą częścią jest funkcja kryterialna oceniająca wartość zagrania, która w niektórych przypadkach musi być oparta o probabilistykę wystąpienia kart u przeciwnika. 

Rozważmy przypadek, w którym algorytm musi podjąć decyzję o zagraniu karty Strażniczki, lub karty Barona. Jest to pierwszy ruch gracza, a w widocznych kartach odrzuconych na starcie są odrzucone karty Króla, Księcia i Pokojówki. Oznacza to, że w talii pozostało 9 kart, a jedna z odrzuconych jest niewidoczna, niemniej jednak ją też trzeba brać pod uwagę. Wyliczenie prawdopodobieństwa \textit{P} jaką kartę ma przeciwnik jest tym momencie proste, jednak trzeba jeszcze wziąć pod uwagę drobny szczegół - czy do liczenia \textit{P} wliczać karty posiadane w ręce. Z jednej strony wydaje się to nielogiczne i może prowadzić do wybierania nieoptymalnych decyzji (co przeczyłoby idei algorytmu zachłannego), z drugiej strony można to potraktować jako element blefu, który jest nieodłączną częścią każdej gry towarzyskiej. W swojej implementacji założyłem absolutną zachłanność algorytmu i karty posiadane na ręce są pomijane w obliczeniach. 
Wobec powyższego, prawdopodobieństwo wystąpienia kart u przeciwnika rozkłada się następująco:


\begin{table}[h]
	\caption{Przykład rozkładu prawdopodobieństwa}
	\centering
		\begin{tabular}{|l|r|}
			\hline
			Karta & $P_{(Karta)}$	\\ \hline
			Strażniczka & 30\% 			\\ \hline
			Kapłan & 20\% 				\\ \hline
			Baron & 10\% 				\\ \hline
			Strażniczka & 10\% 			\\ \hline
			Książę & 10\% 				\\ \hline
			Hrabina & 10\% 				\\ \hline
			Księżniczka & 10\% 			\\ \hline
		\end{tabular}
\end{table}

Zagranie karty Baron oznacza porównanie drugiej karty z kartą przeciwnika. Łatwo policzyć, że w 70\% przypadków skończyłoby to się porażką, a w 30\% nie było by żadnego efektu. Drugim dostępnym ruchem jest zagranie karty Strażniczki, a w jej przypadku najlepszym wyborem jest wytypowanie Kapłana, co daje 20\% szans na zwycięstwo i 80\% szans, że nie nastąpi żaden efekt. Zauważmy, że gdybyśmy wliczali posiadane karty do obliczenia prawdopodobieństwa, wystąpienie Barona i Kapłana byłoby tak samo możliwe. W takich przypadkach algorytm powinien zawsze celować w kartę z wyższym numerem. By formalnie stwierdzić, jaka decyzja powinna zostać podjęta, musimy obliczyć funkcję kryterialną dla dostępnych zagrań i wybrać to zagranie, dla której funkcja przyjmuje wyższy wynik. Przyjmijmy, że funkcja kryterialna wygląda następująco:

\begin{center}
	$F_{(karta)} = 1 + prawdopodobienstwo\_wygranej - prawdopodobienstwo\_przegranej$
\end{center}
Po podstawieniu otrzymujemy:
\begin{center}
 $F_{(Strazniczka)}=1.2$ i $F_{(Baron)} = 0.3$
 \\
 $F_{(Strazniczka)}>F_{(Baron)} => Decyzja=Zagranie_{(Pokojowka + Kaplan)}$ 
 \end{center} 

Najlepszą decyzją w tym wypadku jest zagranie karty Strażniczki z wyborem karty Kapłana. Jak jednak na podstawie powyższego wzoru ocenić zagranie karty Kapłana, Pokojówki lub Króla? Każda z nich wymaga unikalnej funkcji kryterialnej. Mając na uwadze, że w kontekście strategii zachłannej decyzja zawsze powinna być optymalna w ujęciu chwili, ustaliłem następujące warunki oceny zagrania karty:
\begin{itemize}
	\item Strażniczka - ocena zagrania wzrasta gdy pozwala wyeliminować przeciwnika.
	\item Kapłan - w ujęciu chwili jest to karta neutralna i ocena jej zagrania będzie zawsze stała.
	\item Baron - ocena zagrania wzrasta gdy mamy drugą kartę silniejszą niż może mieć przeciwnik.
	\item Pokojówka - podobnie jak w przypadku kapłana, ocena zagrania karty będzie stała.
	\item Książę - ocena wzrasta z prawdopodobieństwem Księżniczki u przeciwnika, lub gdy druga posiadana karta ma mniejszą siłę niż może mieć przeciwnik (wtedy zagrywana na siebie)
	\item Król - premiowane gdy przeciwnik ma Księżniczkę lub Hrabinę.
	\item Hrabina - karta neutralna, jednak musimy ją zagrać w określonych sytuacjach.
	\item Księżniczka - nie może być nigdy wyrzucona.
	\item Dodatkowo, jeśli oba zagrania mają taką samą wartość, powinna być podjęta decyzja o zagraniu karty o niższej wartości.
\end{itemize}

\subsection{Zapis pseudokodem}
Strategię zachłanną $D_n$ najlepiej przedstawić za pomocą pseudokodu:
\begin{algorithmic}[1]
	\Function{$D_n$}{$X_n,s_{n-1}$}
		\ForAll{ rodzajKarty } \Comment Tworzenie tablicy prawdopodobieństwa
			\State $P[rodzajKarty] \gets$  prawdopodobieństwo wystąpienia karty danego rodzaju u przeciwnika	
		\EndFor
		\ForAll{ $x_i$ } \Comment Obliczenie funkcji kryterialnej dla każdego zagrania
				\State K[$x_i$] $\gets$ $F(x_i, P[])$
		\EndFor
		
		\State $ decyzja \gets x_0$ \Comment Szukanie zagrania o najwyższej wycenie
		\ForAll{ $x_i$ } 
			\If {$K[decyzja] < K[x_i]$}
				\State $decyzja \gets x_i$
			\EndIf
		\EndFor

		\State \textbf{return} $decyzja$
	\EndFunction
\end{algorithmic}

Gdzie funkcja kryterialna wygląda następująco:
\begin{algorithmic}[1]
	\Function{$F$}{$x_i,P[]$}	
		\Switch{$x_i$}
			\Case{Strażniczka + Kaplan} \Comment Zagranie Strażniczki na dany typ karty
				\State \textbf{return} $ 1 + P[Kaplan] + 0.08 $
			\EndCase
			\Case{Strażniczka + Baron}
				\State \textbf{return} $ 1 + P[Baron]  + 0.08 $
			\EndCase
			\State ...
			\Case{Strażniczka + Księżniczka}
				\State \textbf{return} $ 1 + P[Ksiezniczka]  + 0.08 $
			\EndCase
			\Case{Kapłan}
				\State \textbf{return} $ 1 + 0.07 $
			\EndCase
			\Case{Baron}	\Comment Kryterium zależy od wartości W() drugiej posiadanej karty (DK)
				\State $ szansePrzegranej \gets 0$ 
				\State $ szanseWygranej \gets 0$ 
				\ForAll {$typKarty$}
					\If {$ W(DK) < W(typKarty) $}  
						\State $szansePrzegranej \gets szansePrzegranej + P[typKarty]$ 
					\ElsIf {$ W(DK) > W(typKarty) $}
					\State $szanseWygranej \gets szanseWygranej + P[typKarty]$ 
					\EndIf
				\EndFor
				\State \textbf{return} $ 1 - szansePrzegranej + szanseWygranej + 0.06 $
			\EndCase
			\Case{Pokojówka}
				\State \textbf{return} $ 1 + 0.05 $
			\EndCase
			\Case{Książe na przeciwnika}
				\State \textbf{return} $ 1 + P[Ksiezniczka] + 0.04 $
			\EndCase
			\Case{Książe na siebie} 
				\State \textbf{return} $ 1 + 0.04 $
			\EndCase
			\Case{Król}
				\State \textbf{return} $ 1 + P[Ksiezniczka] + P[Hrabina] + 0.03 $
			\EndCase
			\Case{Hrabina, gdy druga posiadana karta to Król lub Książe}
				\State \textbf{return} $ 10 + 0.02 $
			\EndCase
			\Case{Hrabina}
				\State \textbf{return} $ 1 + 0.02 $
			\EndCase
			\Case{Księżniczka}
				\State \textbf{return} $ 0 $
			\EndCase
		\EndSwitch
	\EndFunction
\end{algorithmic}

		
\section{Algorytm Monte Carlo Tree Search}
\label{sec:algMCTS}
\chapter{Implementacja}
\label{cha:rozdz4}

\section{Analiza wymagań}

\section{Koncepcja wykonania}

\section{Wykorzystane technologie}

\section{Diagramy}

\section{Prezentacja systemu}

\section{Problemy napotkane w trakcie realizacji}
\chapter{Rezultaty}
\label{cha:rozdz5}

W tym rozdziale zamieszczam rezultaty symulacji gier z użyciem opisanych wcześniej algorytmów. Dla każdego z nich przedstawiam średni procent zwycięstw oraz zagrania, które najczęściej kończyły grę. Na początku porównuję każdy z algorytmów do losowego, podając szczegółowe statystyki. Następnie porównuję pozostałe algorytmy względem siebie (z uwzględnieniem kolejności graczy), skupiając się na charakterystycznych danych. Rozdział kończę opisaniem wniosków z działania algorytmów. Dane oparte są na próbie 100 tysięcy symulacji, z wyjątkiem tych, gdzie porównywany jest algorytm MCTS. W tych przypadkach, względu na czas trwania jednej rozgrywki, ograniczyłem próbę do 1 tysiąca.


\section{Losowy versus Losowy}

\begin{figure}[H]
	\centering
	\includegraphics[width=\textwidth]{Resources/rvr/win.PNG}
	\caption{Wykres kołowy zwycięstw i remisów} 
	\label{fig:rvrwin}
\end{figure}

Powyższy wykres na rys. \ref{fig:rvrwin} wskazuje na pewną przewagę gracza rozpoczynającego. Co ciekawe, nie dotyczy to gier zakończonych porównaniem siły kart. Tutaj statystyki w logach wskazują na 6309 zwycięstwa pierwszego gracza i 6158 drugiego.

\begin{figure}[H]
	\centering
	\includegraphics[width=\textwidth]{Resources/rvr/roundwin.PNG}
	\caption{Wykres wygranych w danej rundzie} 
	\label{fig:rvrroundwin}
\end{figure}

Na rys. \ref{fig:rvrroundwin} widać wyraźną tendencję kończenia się gry w pierwszych dwóch rundach i w przedostatniej. 

\clearpage
\begin{figure}[H]
	\centering
	\includegraphics[width=\textwidth]{Resources/rvr/decision.PNG}
	\caption{Wykres zwycięskich zagrań} 
	\label{fig:rvrdecision}
\end{figure} 

\begin{figure}[H]
	\centering
	\includegraphics[width=\textwidth]{Resources/rvr/guarddecision.PNG}
	\caption{Wykres szczegółowy zwycięskich zagrań karty Strażniczki} 
	\label{fig:rvrguarddecision}
\end{figure}

Z dwóch powyższych wykresów (rys. \ref{fig:rvrwin} i rys. \ref{fig:rvrguarddecision}) wynika, że zdecydowanie najczęstszym wygrywającym ruchem jest zagranie karty Baron. Około połowę mniej zwycięstw jest w wyniku zagrania karty Strażniczki. Warty uwagi jest fakt, że im bliżej końca gry tym częściej zwycięskim zagraniem jest zagranie karty Księcia na przeciwnika.

\section{Zachłanny versus Losowy}

\begin{figure}[H]
	\centering
	\includegraphics[width=\textwidth]{Resources/gvr/win.PNG}
	\caption{Wykres kołowy zwycięstw i remisów} 
	\label{fig:gvrwin}
\end{figure}

Powyżej (rys. \ref{fig:gvrwin}) widać zauważalną przewagę algorytmu zachłannego nad losowym. Stosunek gier zakończonych porównaniem wynosi 3965 do 2621 na korzyść zachłannego.

\begin{figure}[H]
	\centering
	\includegraphics[width=\textwidth]{Resources/gvr/roundwin.PNG}
	\caption{Wykres wygranych w danej rundzie} 
	\label{fig:gvrroundwin}
\end{figure}

Na rys. \ref{fig:gvrroundwin} widać różnicę w porównaniu do symulacji gdzie obaj gracze grają losowo. Nie ma tak wyraźnego wzrostu gier skończonych w przedostatniej rundzie. Wynika to ze znacznie mniejszej ilości gier zakończonych remisem.

\clearpage
\begin{figure}[H]
	\centering
	\includegraphics[width=\textwidth]{Resources/gvr/decision.PNG}
	\caption{Wykres zwycięskich zagrań} 
	\label{fig:gvrdecision}
\end{figure} 

\begin{figure}[H]
	\centering
	\includegraphics[width=\textwidth]{Resources/gvr/guarddecision.PNG}
	\caption{Wykres szczegółowy zwycięskich zagrań karty Strażniczki} 
	\label{fig:gvrguarddecision}
\end{figure}

Porównanie wykresów na rysunkach \ref{fig:rvrguarddecision} oraz \ref{fig:gvrguarddecision} pokazują ogromny wzrost znaczenia zagrania karty Strażniczki z wyborem karty Książę. Algorytm zachłanny osiąga zwycięstwo zagrywając Strażniczkę niemal tak samo często co przy zagraniu Barona.

\section{Minimaksowy versus Losowy}

\begin{figure}[H]
	\centering
	\includegraphics[width=\textwidth]{Resources/mmvr/win.PNG}
	\caption{Wykres kołowy zwycięstw i remisów} 
	\label{fig:mmvrwin}
\end{figure}

Powyższy wykres (rys. \ref{fig:mmvrwin}) pokazuje bardzo podobne wyniki jak w przypadku algorytmu zachłannego. Stosunek gier zakończonych porównaniem wynosi 4023 do 2610 na korzyść minimaksowego.

\begin{figure}[H]
	\centering
	\includegraphics[width=\textwidth]{Resources/mmvr/roundwin.PNG}
	\caption{Wykres wygranych w danej rundzie} 
	\label{fig:mmvrroundwin}
\end{figure}

Również w przypadku wykresu na rys. \ref{fig:mmvrroundwin} wyniki są bardzo do algorytmu zachłannego.

\begin{figure}[H]
	\centering
	\includegraphics[width=\textwidth]{Resources/mmvr/decision.PNG}
	\caption{Wykres zwycięskich zagrań} 
	\label{fig:mmvrdecision}
\end{figure} 

Wykres na rysunku \ref{fig:mmvrdecision} w porównaniu do \ref{fig:gvrdecision} pozwala zauważyć niewielki wzrost zwycięstw w drugiej rundzie i spadek w trzeciej.

\begin{figure}[H]
	\centering
	\includegraphics[width=\textwidth]{Resources/mmvr/guarddecision.PNG}
	\caption{Wykres szczegółowy zwycięskich zagrań karty Strażniczki} 
	\label{fig:mmvrguarddecision}
\end{figure}

Na rys. \ref{fig:mmvrguarddecision} widać, że statystyki zagrania karty Strażniczki są niemal takie same jak w przypadku algorytmu zachłannego.

\section{MCTS versus Losowy}

Dane z symulacji algorytmu MCTS opierają się na próbie 1000 gier, ponieważ większa próba znacząco zwiększa czas oczekiwania. Wynika to z charakteru algorytmu, który działa przez zadany czas. W przeprowadzonych symulacjach czas działania algorytmu ustawiony był na 100ms.

\begin{figure}[H]
	\centering
	\includegraphics[width=\textwidth]{Resources/mctsvr/win.PNG}
	\caption{Wykres kołowy zwycięstw i remisów} 
	\label{fig:mctsvrwin}
\end{figure}

Powyższy wykres (rys. \ref{fig:mctsvrwin}) pokazuje w tym wypadku wyraźną przewagę algorytmu losowego nad algorytmem Monte Carlo Tree Search. Powody tego stanu rzeczy opisane są we wnioskach.

\begin{figure}[H]
	\centering
	\includegraphics[width=\textwidth]{Resources/mctsvr/roundwin.PNG}
	\caption{Wykres wygranych w danej rundzie} 
	\label{fig:mctsvrroundwin}
\end{figure}

Wykres na rys. \ref{fig:mctsvrroundwin} pozwala zauważyć pewną charakterystykę podobną do symulacji Losowy versus Losowy - średnia liczba zwycięstw wzrasta w ostatnich rundach.

\begin{figure}[H]
	\centering
	\includegraphics[width=\textwidth]{Resources/mctsvr/decision.PNG}
	\caption{Wykres zwycięskich zagrań} 
	\label{fig:mctsvrdecision}
\end{figure} 

\begin{figure}[H]
	\centering
	\includegraphics[width=\textwidth]{Resources/mctsvr/guarddecision.PNG}
	\caption{Wykres szczegółowy zwycięskich zagrań karty Strażniczki} 
	\label{fig:mctsvrguarddecision}
\end{figure}

Wykresy na rysunkach \ref{fig:mctsvrdecision} i \ref{fig:mctsvrguarddecision} pokazują zbliżone charakterystyki do algorytmu losowego.

\section{Minimaksowy versus Zachłanny}

\begin{figure}[H]
	\centering
	\includegraphics[width=\textwidth]{Resources/mmvg/win.PNG}
	\caption{Wykres kołowy zwycięstw i remisów; Minimaksowy pierwszym graczem} 
	\label{fig:mmvgwin}
\end{figure}

\begin{figure}[H]
	\centering
	\includegraphics[width=\textwidth]{Resources/gvmm/win.PNG}
	\caption{Wykres kołowy zwycięstw i remisów; Zachłanny pierwszym graczem} 
	\label{fig:gvmmwin}
\end{figure}

Powyższe wykresy (rys. \ref{fig:mmvgwin} i \ref{fig:gvmmwin}) pokazują niewielką przewagę algorytmu minimaksowego nad zachłannym.


\begin{figure}[H]
	\centering
	\includegraphics[width=\textwidth]{Resources/mmvg/decision.PNG}
	\caption{Wykres zwycięskich zagrań} 
	\label{fig:mmvgdecision}
\end{figure} 

\begin{figure}[H]
	\centering
	\includegraphics[width=\textwidth]{Resources/gvmm/decision.PNG}
	\caption{Wykres zwycięskich zagrań} 
	\label{fig:gvmmdecision}
\end{figure} 

\begin{figure}[H]
	\centering
	\includegraphics[width=\textwidth]{Resources/mmvg/guarddecision.PNG}
	\caption{Wykres szczegółowy zwycięskich zagrań karty Strażniczki} 
	\label{fig:mmvgguarddecision}
\end{figure}

\begin{figure}[H]
	\centering
	\includegraphics[width=\textwidth]{Resources/gvmm/guarddecision.PNG}
	\caption{Wykres szczegółowy zwycięskich zagrań karty Strażniczki} 
	\label{fig:gvmmguarddecision}
\end{figure}


Oba algorytmy wykazują duże podobieństwa, a różnice między nimi są takie same jak podczas konfrontacji z algorytmem losowym.
\clearpage
\section{Zachłanny versus MCTS}

\begin{figure}[H]
	\centering
	\includegraphics[width=\textwidth]{Resources/gvmcts/win.PNG}
	\caption{Wykres kołowy zwycięstw i remisów; Zachłanny pierwszym graczem} 
	\label{fig:gvmctswin}
\end{figure}

\begin{figure}[H]
	\centering
	\includegraphics[width=\textwidth]{Resources/mctsvg/win.PNG}
	\caption{Wykres kołowy zwycięstw i remisów; MCTS pierwszym graczem} 
	\label{fig:mctsvgwin}
\end{figure}

Powyższe wykresy (rys. \ref{fig:gvmcts} i \ref{fig:mctsvg}) pokazują bardzo wyraźną przewagę algorytmu zachłannego nad MCTS.

\begin{figure}[H]
	\centering
	\includegraphics[width=\textwidth]{Resources/gvmcts/decision.PNG}
	\caption{Wykres zwycięskich zagrań; Zachłanny pierwszym graczem} 
	\label{fig:gvmctsdecision}
\end{figure} 

\begin{figure}[H]
	\centering
	\includegraphics[width=\textwidth]{Resources/mctsvg/decision.PNG}
	\caption{Wykres zwycięskich zagrań; MCTS pierwszym graczem} 
	\label{fig:mctsvgdecision}
\end{figure} 

Powyższe wykresy (rys. \ref{fig:gvmctsdecision} i \ref{fig:mctsvgdecision}) wykazują pewną ciekawą tendencję algorytmu MCTS do wygrywania po zagraniu karty Barona w 2 rundzie, w przeciwieństwie do algorytmu zachłannego. Może to wynikać z tego, że MCTS częściej zatrzymuje w ręce kartę o wyższym numerze.

\begin{figure}[H]
	\centering
	\includegraphics[width=\textwidth]{Resources/gvmcts/guarddecision.PNG}
	\caption{Wykres szczegółowy zwycięskich zagrań karty Strażniczki} 
	\label{fig:gvmctguarddecision}
\end{figure}

\begin{figure}[H]
	\centering
	\includegraphics[width=\textwidth]{Resources/mctsvg/guarddecision.PNG}
	\caption{Wykres szczegółowy zwycięskich zagrań karty Strażniczki} 
	\label{fig:mctsvgguarddecision}
\end{figure}

Na powyższych wykresach (rys. \ref{fig:gvmctguarddecision} i \ref{fig:mctsvgguarddecision}) można zauważyć zupełnie inne tendencje do zagrywania karty Strażniczki. Algorytm Zachłanny preferuje wybór Księcia w pierwszych rundach i Księżniczki w późniejszych rundach, natomiast MCTS zamiast Księcia częściej wybiera Kapłana.

\section{Wnioski}
Ogólne:
Zdecydowana większość gier kończy się w pierwszych 3 rundach, głównie poprzez zagranie kart Strażniczki i Barona. Wynika z tego również, że gracz rozpoczynający ma znaczną przewagę nad drugim graczem. W kolejnych rundach średnia ilość zwycięstw maleje, i następnie rośnie w dwóch ostatnich. Wynika to z dwóch zjawisk: po pierwsze, przy końcu gry gracze wyzbyli się już kart ofensywnych, czyli Strażniczki, Barona i Księcia, wobec czego dochodzi do porównania sił kart. Po drugie, jeśli karty ofensywne pozostały grze, znacznie wzrasta skuteczność ich użycia, co widać po częstości zagrań Strażniczki ze wskazaniem na Księżniczkę, bądź Księcia ze wskazaniem na przeciwnika (spodziewając się, że ma Księżniczkę). Ze statystyk można wywnioskować, że gra zdecydowanie sprzyja ofensywnym zagraniom.

Szczegółowe:
Algorytm losowy nie wymaga głębokiej analizy - niemniej jednak warty uwagi jest fakt, że jest bardziej skuteczny niż algorytm MCTS.

Algorytm zachłanny osiąga wysokie wyniki w porównaniu z innymi algorytmami i niewiele niższe niż algorytm minimaksowy. Wynika to ze wspomnianego faworyzowania przez grę zagrań, które jak najszybciej zakończą grę. 

Algorytm minimaksowy osiąga wyniki tylko niewiele lepsze niż algorym zachłanny, pomimo znacznie dłuższego czasu działania. Warto pamiętać, że w przedstawionym wariancie dokonuje on tylko przeszukania drzewa rozwiązań do 1 ruchu w przód. Algorytm ten osiąga przewagę nad zachłannym w rundach środkowych, jednak liczba zwycięstw z tego uzyskanych jest niewielka.

Monte Carlo Tree Search okazał się być największym rozczarowaniem, szczególnie, że jest mniej efektywny od algorytmu losowego. Przyczyną tego stanu rzeczy jest moje błędne założenie, że negatywny wpływ elementu losowego na algorytm może zostać zniwelowany przez ilość symulacji przeprowadzanych przez MCTS. Niestety, implementacja algorytmu w formie podstawowej sprawia, że wpada on w pułapkę już podczas pierwszego kroku. Wynika to z tego, że gracz w danym momencie gry nie zna całego jej stanu, lecz zbiór informacyjny \textit{I}, w związku z tym w korzeniu drzewa zawarty jest jeden wylosowany stan ze zbioru informacyjnego. Biorąc pod uwagę ilość stanów w zbiorze informacyjnym, szansa, że zostanie wylosowany ten faktyczny, jest niewielka. Każda symulacja wykonana na błędnie założonym stanie początkowym w korzeniu drzewa algorytmu MCTS będzie prowadzić do błędnych wniosków. W efekcie MCTS osiąga znacznie gorsze wyniki niż algorytm losowy. Można by to podsumować potocznym stwierdzeniem, że "świadomość braku wiedzy jest lepsza od nieświadomości życia w błędzie". 
\chapter{Podsumowanie}
\label{cha:rozdz6}

Celem mojej pracy była analiza i porównanie efektywności wybranych algorytmów w podejmowaniu decyzji w grze ,,Love Letter''. Podstawowym założeniem było nie przeszukiwanie całego drzewa rozwiązań, lecz jedynie najbliższy poziom, bądź posłużenie się heurystyką.

Opisałem zasady gry ,,Love Letter'', a następnie oszacowałem ilość możliwych rozwiązań. Na tej podstawie przedstawiłem ją jako problem optymalizacyjny, który stał się podstawą do porównania efektywności algorytmów. Model gry zapisałem w postaci gry ekstensywnej opierając się na Teorii Gier.

Do analizy i porównania wybrałem następujące algorytmy: losowy, zachłanny, minimaksowy oraz Monte Carlo Tree Search, który jest algorytmem heurystycznym. Sposób działania każdego z nich został opisany i podałem przykłady ich wykorzystania w praktyce. Następnie zaprezentowałem pomysł ich użycia do podejmowania decyzji w grze ,,Love Letter''.

Kolejno zaprezentowałem założenia programu, w którym zaimplementowane zostały zasady gry oraz wspomniane algorytmy. Przedstawiłem analizę wymagań oraz diagramy objaśniające strukturę programu. W analizie post implementacyjnej wskazałem główny problem związany z napisaniem programu, jakim okazał się niedostateczny poziom abstrakcji, znacznie zwiększający objętość kodu i tym samym ryzyko popełnienia błędu.

Następnie zaprezentowałem wyniki przeprowadzonych przeze mnie symulacji gier. W analizie skupiłem się na ilości zwycięstw danego algorytmu oraz najczęściej wygrywających zagraniach. Ważnym wnioskiem było wskazanie najskuteczniejszego z analizowanych algorytmów, jakim okazała się moja implementacja algorytmu minimaksowego. Z pewnością jego skuteczność mogłaby być wyższa, jednak z racji charakteru gry, który preferuje szybkie kończenie rund ponad zagrywki przedłużające grę, czas działania algorytmu rósł by zdecydowanie szybciej niż jego efektywność. 

Moim największym zaskoczeniem były wyniki algorytmu MCTS, które były gorsze nawet od wyników algorytmu losowego. Taki stan rzeczy wynika z błędnie przyjętych na początku założeń, że podstawowa wersja tego algorytmu będzie w stanie osiągać dobre wyniki w tej grze. Mimo to uważam, że po odpowiedniej modyfikacji polegającej na zamianie stanów znajdujących się w węzłach na zbiory informacyjne, byłby on w stanie osiągać wyniki równe, a nawet lepsze, od algorytmu minimaksowego.

\bibliographystyle{alpha}

\begin{thebibliography}{1}

\bibitem{}
\label{bib:loveLetterGame}
Alderac Entertainment Group,
\newblock {\em Love Letter},
\newblock 2012.

\bibitem{}
\label{bib:loveLetterWebsite}
Alderac Entertainment Group,
\newblock {\em Love Letter},
\newblock http://www.alderac.com/tempest/love-letter, 2016-02-08.

\bibitem{}
\label{bib:tabliceMatematyczne}
Alicja Cewe, Halina Nahorska, Irena Pancer,
\newblock {\em Tablice matematyczne},
\newblock Wydawnictwo Podkowa, Gdańsk 2002,
\newblock rozdział {\em Kombinatoryka}.

\bibitem{}
\label{bib:wiki_ProblemOptymalizacyjny}
\newblock {\em Wikipedia}, hasło {\em Problem optymalizacyjny}.
\newblock https://pl.wikipedia.org/wiki/Problem\_optymalizacyjny, 2016-02-21.

\bibitem{}
\label{bib:wiki_StrategiaTeoriaGier}
\newblock {\em Wikipedia}, hasło {\em Strategia w teorii gier}.
\newblock https://pl.wikipedia.org/wiki/Strategia\_mieszana, 2016-04-10.

\bibitem{}
\label{bib:matematyczneModeleKonfliktu_klasyfikacja}
Andrzej Z. Grzybowski
\newblock {\em Matematyczne modele konfliktu. Wykłady z Teorii Gier i Decyzji},
\newblock Wydawnictwo Politechniki Częstochowskiej, Częstochowa 2012,
\newblock rozdział {\em 1.1 Klasyfikacja problemów decyzyjnych}.

\bibitem{}
\label{bib:matematyczneModeleKonfliktu_graEkstensywna}
Andrzej Z. Grzybowski
\newblock {\em Matematyczne modele konfliktu. Wykłady z Teorii Gier i Decyzji},
\newblock Wydawnictwo Politechniki Częstochowskiej, Częstochowa 2012,
\newblock rozdział {\em 2.1 Gry w postaci ekstensywnej}.

\bibitem{}
\label{bib:wiki_drzewo}
\newblock {\em Wikipedia}, hasło {\em Drzewo (matematyka)}.
\newblock https://pl.wikipedia.org/wiki/Drzewo\_(matematyka), 2016-06-01.

\bibitem{}
\label{bib:wiki_TeoriaDecyzji}
\newblock {\em Wikipedia}, hasło {\em Teoria decyzji}.
\newblock https://pl.wikipedia.org/wiki/Teoria\_decyzji, 2016-04-10.

\bibitem{}
\label{bib:algorytmy_zachlanny}
Sanjoy Dasgupta, Christos Papadimitriou, Umesh Vazirani,
\newblock {\em Algorytmy},
\newblock Wydawnictwo Naukowe PWN, Warszawa 2012,
\newblock rozdział {\em Algorytmy Zachłanne}, s. 133.

\bibitem{}
\label{bib:wiki_minMax}
\newblock {\em Wikipedia}, hasło {\em Algorytm min-max}.
\newblock https://pl.wikipedia.org/wiki/Algorytm\_min-max, 2016-04-24.


\bibitem{}
\label{bib:minMax_warcaby}
\newblock {\em Funkcja oceniająca do algorytmu minimaksu w grze warcaby},
\newblock http://sequoia.ict.pwr.wroc.pl/~witold/aiarr/2009\_projekty/warcaby/, 2016-05-04.

\bibitem{}
\label{bib:wazniak_minMax}
\newblock {\em Studia Informatyczne}, Sztuczna inteligencja/SI Moduł 8 - Gry dwuosobowe.
\newblock http://wazniak.mimuw.edu.pl/index.php?title=Sztuczna\_inteligencja\/SI\_Moduł\_8\_-\_Gry\_dwuosobowe, 2016-04-24.

\bibitem{}
\label{bib:mcts_hex}
Broderick Arneson, Ryan Hayward, Philip Henderson,
\newblock {\em MoHex Wins Hex Tournament},
\newblock ICGA Journal Vol. 32 No. 2 s. 114–116, Czerwiec 2009

\bibitem{}
\label{bib:wiki_mcts}
\newblock {\em Wikipedia}, hasło {\em Monte-Carlo Tree Search}.
\newblock https://pl.wikipedia.org/wiki/Monte-Carlo\_Tree\_Search, 2016-05-04.

\bibitem{}
\label{bib:mcts_wprowadzenie}
\newblock {\em Introduction to Monte Carlo Tree Search},
\newblock https://jeffbradberry.com/posts/2015/09/intro-to-monte-carlo-tree-search/, 2016-05-04.

\bibitem{}
\label{bib:mcts_opis}
Guillaume Chaslot, Sander Bakkes, Istvan Szita, Pieter Spronck,
\newblock {\em Monte-Carlo Tree Search: A New Framework for Game AI},
\newblock Universiteit Maastricht / MICC.
\newblock P.O. Box 616, NL-6200 MD Maastricht, The Netherlands

\bibitem{}
\label{bib:mcts_totalWar}
\newblock {\em Monte-Carlo Tree Search in TOTAL WAR: ROME II’s Campaign AI},
\newblock http://aigamedev.com/open/coverage/mcts-rome-ii/, 2016-05-04.

\bibitem{}
\label{bib:mcts_osadnicy}
Istvan Szita, Guillaume Chaslot, Pieter Spronck,
\newblock {\em Monte-Carlo Tree Search in Settlers of Catan} 
\newblock Volume 6048 of the series Lecture Notes in Computer Science, s. 21-32, 2010.

\bibitem{}
\label{bib:mcts_uct}
\newblock {\em Monte Carlo Tree Search}, sekcja {\em About},
\newblock http://www.cameronius.com/research/mcts/about/index.html, 2016-05-05.

\bibitem{}
\label{bib:analiza_wymagan}
Krzysztof Sacha
\newblock {\em Inżynieria oprogramowania}
\newblock Wydawnictwo Naukowe PWN, Warszawa 2014,
\newblock rozdział {\em Inżynieria wymagań}, s. 50.

\bibitem{}
\label{bib:wiki_wierszPolecen}
\newblock {\em Wikipedia}, hasło {\em Wiersz poleceń}.
\newblock https://pl.wikipedia.org/wiki/Wiersz\_poleceń, 2016-05-08.

\bibitem{}
\label{bib:java}
Oracle Corporation
\newblock {\em Java}
\newblock http://www.oracle.com/technetwork/java/javase/downloads/index.html, 2016-05-08.

\bibitem{}
\label{bib:analiza_wymagan}
Krzysztof Sacha
\newblock {\em Inżynieria oprogramowania}
\newblock Wydawnictwo Naukowe PWN, Warszawa 2014,
\newblock rozdział {\em Metodyka zwinna}, s. 334.

\bibitem{}
\label{bib:wiki_waterfall}
\newblock {\em Wikipedia}, hasło {\em Model kaskadowy}.
\newblock https://pl.wikipedia.org/wiki/Model\_kaskadowy, 2016-05-08.
	
\bibitem{}
\label{bib:intellij}
JetBrains
\newblock {\em IntelliJ IDEA}
\newblock https://www.jetbrains.com/idea/, 2016-05-08.

\bibitem{}
\label{bib:czysty_kod}
Robert C. Martin
\newblock {\em Czysty Kod}
\newblock Wydawnictwo Helion, Gliwice 2014


\bibitem{}
\label{bib:czysty_kod_cytat_booch}
Robert C. Martin
\newblock {\em Czysty Kod}
\newblock Wydawnictwo Helion, Gliwice 2014,
\newblock rozdział 1. {\em Czysty Kod}, s. 30.

\bibitem{}
\label{bib:wiki_SOLID}
\newblock {\em Wikipedia}, hasło {\em SOLID (programowanie obiektowe)}.
\newblock https://pl.wikipedia.org/wiki/SOLID\_(programowanie\_obiektowe), 2016-06-28.

\end{thebibliography}
\end{document}
