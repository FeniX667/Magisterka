\chapter{Wprowadzenie}
\label{cha:rozdz1}
%---------------------------------------------------------------------------
\section{Przedmiot pracy}
\label{sec:przedmiotPracy}

Od kilku lat coraz większą popularnością cieszą się wszelkiego rodzaju gry planszowe i karciane. Przyciągają nie tylko coraz lepszą oprawą graficzną, lecz również ciekawą mechaniką pozwalającą na stosowanie różnych taktyk. Z tego powodu stanowią szerokie pole do testowania algorytmów optymalizujących dostępne ruchy tak, by zapewnić zwycięstwo.

Są również gry, w których kluczową rolę odgrywa tak zwana 'intuicja'. Obliczenie całego drzewa dostępnych ruchów jest zbyt skomplikowane i decyzja musi zostać podjęta na podstawie niepełnych danych. Przykładem takiej gry jest "Love Letter", gra niezbyt skomplikowana, jednak zawierająca dużo interakcji i możliwych ścieżek rozwoju sytuacji.

Inspirując się wyżej wymienioną grą, w poniższej pracy porównuję trzy algorytmy podejmowania decyzji w grze:
\begin{itemize}
\item
probabilistyczny - który będzie podejmował decyzję w sposób losowy na podstawie prawdopodobieństwa wystąpień kart,
\item
zachłanny - który będzie wybierał zawsze najbardziej prawdopodobny scenariusz,
\item
Monte Carlo Tree Search - algorytm heurystyczny, który podejmowane decyzje opiera na symulacjach.
\end{itemize} 

%---------------------------------------------------------------------------
\section{Cel pracy}
\label{sec:zawartoscPracy}

\section{Zawartość pracy}
\label{sec:zawartoscPracy}

Rozdział~\ref{cha:rozdz1} jest wprowadzeniem definiującym przedmiot pracy. W rozdziale~\ref{cha:rozdz2} opisana została gra karciana wraz z problemem który przedstawia. Rozdział~\ref{cha:rozdz3} przedstawia trzy proponowane algorytmy rozwiązujące problem. W rozdziale~\ref{cha:rozdz4} zebrane są poszczególne etapy tworzenia aplikacji:
\begin{itemize}
\item
analiza wymagań,
\item
koncepcja wykonania,
\item
wykorzystane technologie,
\item
diagramy,
\item
prezentacja systemu.
\end{itemize} 

W rozdziale~\ref{cha:rozdz5} zebrane są statystyki związane z implementacją algortytmów. Rozdział~\ref{cha:rozdz6} stanowi podsumowanie pracy.