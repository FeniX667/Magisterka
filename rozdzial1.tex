\chapter{Wprowadzenie}
\label{cha:rozdz1}
%---------------------------------------------------------------------------
\section{Przedmiot pracy}
\label{sec:przedmiotPracy}

Od kilku lat coraz większą popularnością cieszą się wszelkiego rodzaju gry planszowe i karciane. Ze względu na ich różnorodną mechanikę, mogą być dobrą podstawą do porównania i analizy działania różnych algorytmów. Gdy grając w jedną z nich, ,,Love Letter'', mój kolega stwierdził, że ,,ta gra to tylko los'', postanowiłem sprawdzić to w praktyce, implementując kilka wybranych algorytmów i sprawdzając ich działanie właśnie na podstawie tej gry. 
Algorytmy zostały przeze mnie wybrane w taki sposób, by mimo deterministycznej natury przypominać intuicyjne zachowanie człowieka w sytuacji dużej niepewności. Niektórzy podejmują decyzje losowo, inni wybierają te zagrania, które najszybciej prowadzą do wygranej, a część stara się minimalizować ewentualne straty. Oprócz algorytmów deterministycznych wybrałem również jeden heurystyczny, który poprzez swoją niejednoznaczność w podejmowaniu wyborów najbardziej przypomina ludzkie zachowanie.

%---------------------------------------------------------------------------
\section{Cel pracy}
\label{sec:zawartoscPracy}
Celem tej pracy jest implementacja programistyczna gry oraz kilku różnych algorytmów, których działanie można przeanalizować opierając się na statystykach zdobywanych w symulacji. Następnie, spośród wybranych algorytmów, zostanie wybrany algorytm najlepiej spełniający swoje zadanie - czyli wygrywanie gier.

\section{Zawartość pracy}
\label{sec:zawartoscPracy}
Rozdział \ref{cha:rozdz1} jest wprowadzeniem definiującym przedmiot pracy, jej cel oraz zawartość. W rozdziale \ref{cha:rozdz2} opisana została gra karciana ,,Love Letter'' wraz z zasadami. Na ich podstawie oszacowałem ilość możliwych rozwiązań, a następnie dokonałem analizy problemu i zdefiniowałem jego model matematyczny w postaci gry ekstensywnej. Rozdział \ref{cha:rozdz3} przedstawia proponowane przeze mnie algorytmy rozwiązujące przedstawiony problem, kolejno: algorytm losowy, algorytm zachłanny, algorytm minimaksowy oraz algorytm Monte Carlo Tree Search. Dla każdego z nich opisany został sposób działania, wykorzystanie w praktyce, oraz mój pomysł na użycie do rozwiązania zadanego problemu. W rozdziale \ref{cha:rozdz4} przedstawiłem założenia programu, w którym implementowana jest gra i algorytmy. Następnie zaprezentowane są poszczególne etapy tworzenia aplikacji, począwszy od analizy wymagań, poprzez koncepcję wykonania, prezentację diagramów UML oraz listingów, a skończywszy na prezentacji programu. W rozdziale \ref{cha:rozdz5} przedstawiłem wyniki eksperymentów symulacyjnych, w których brały udział implementowane algorytmy. Statystyki przedstawione zostały na wykresach wraz z komentarzem. Następnie zaprezentowałem wnioski na temat gry oraz poszczególnych algorytmów. W rozdziale \ref{cha:rozdz6} przedstawiłem podsumowanie wykonanych czynności, wnioski z eksperymentów oraz osiągnięty cel pracy.