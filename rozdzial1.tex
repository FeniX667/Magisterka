\chapter{Wprowadzenie}
\label{cha:rozdz1}
%---------------------------------------------------------------------------
\section{Przedmiot pracy}
\label{sec:przedmiotPracy}

Od kilku lat coraz większą popularnością cieszą się wszelkiego rodzaju gry planszowe i karciane. Ze względu na ich różnorodną mechanikę, mogą być dobrą podstawą do porównania i analizy działania różnych algorytmów. Gdy grając w jedną z nich, ,,Love Letter'', mój kolega stwierdził, że ,,ta gra to tylko los'', postanowiłem sprawdzić to w praktyce, implementując kilka wybranych algorytmów i sprawdzając ich działanie właśnie na podstawie tej gry.

%---------------------------------------------------------------------------
\section{Cel pracy}
\label{sec:zawartoscPracy}
Celem tej pracy jest implementacja programistyczna gry oraz kilku różnych algorytmów, których działanie można przeanalizować opierając się na statystykach zdobywanych w symulacji. Następnie, spośród wybranych algorytmów, zostanie wybrany algorytm najlepiej spełniający swoje zadanie - czyli wygrywanie gier.

\section{Zawartość pracy}
\label{sec:zawartoscPracy}
Rozdział \ref{cha:rozdz1} jest wprowadzeniem definiującym przedmiot pracy. W rozdziale \ref{cha:rozdz2} opisana została gra karciana ,,Love Letter'' wraz z zasadami. Na ich podstawie dokonuję analizy problemu i definiuję jego model matematyczny. Rozdział \ref{cha:rozdz3} przedstawia proponowane przeze mnie algorytmy rozwiązujące problem. Dla każdego z nich opisany został sposób działania, wykorzystanie w praktyce, oraz mój pomysł na użycie do rozwiązania zadanego problemu. W rozdziale \ref{cha:rozdz4} zebrane są poszczególne etapy tworzenia aplikacji, począwszy od analizy wymagań, poprzez koncepcję wykonania, prezentację diagramów UML oraz listingów, a skończywszy na prezentacji programu. W rozdziale \ref{cha:rozdz5} przedstawione są statystyki związane z przeprowadzonymi symulacjami oraz wnioski z działania algorytmów. Rozdział \ref{cha:rozdz6} stanowi podsumowanie pracy.