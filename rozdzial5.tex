\chapter{Rezultaty}
\label{cha:rozdz5}

W tym rozdziale zamieszczam rezultaty eksperymentów symulacyjnych z użyciem opisanych wcześniej algorytmów. Dla każdego z nich przedstawiam średni procent zwycięstw oraz zagrania, które najczęściej kończyły grę. Początkowo prezentuję statystyki z eksperymentów w których każdy z algorytmów gra z algorytmem losowym, będącym zawsze jest drugim graczem. Następne prezentuję wyniki z eksperymentów, w których biorą udział algorytm zachłanny, algorytm minimaksowy i algorytm MCTS. W tych przypadkach eksperyment ustawiony jest tak, by każdy z algorytmów był naprzemiennie pierwszym i drugim graczem. Rozdział kończę opisaniem wniosków ogólnych oraz z działania poszczególnych algorytmów.
W każdym porównaniu prezentuję statystyki oparte na próbie 100 tysięcy  symulacji, z wyjątkiem tych, gdzie porównywany jest algorytm MCTS. Dla tych przypadków, ze względu na czas trwania jednej rozgrywki, ograniczyłem próbę do 1 tysiąca.

\section{Algorytm losowy versus Algorytm Losowy}

\begin{figure}[H]
	\centering
	\includegraphics[width=\textwidth]{Resources/RVsR/RVsRwin.PNG}
	\caption{Wykres kołowy zwycięstw i remisów} 
	\label{fig:RVsRwin}
\end{figure}

Powyższy wykres na rys. \ref{fig:RVsRwin} wskazuje na pewną przewagę gracza rozpoczynającego.

\begin{figure}[H]
	\centering
	\includegraphics[width=\textwidth]{Resources/RVsR/RVsRroundwin.PNG}
	\caption{Wykres wygranych w danej rundzie} 
	\label{fig:RVsRroundwin}
\end{figure}

Na rys. \ref{fig:RVsRroundwin} widać wyraźną tendencję do spadku zakończeń gry wraz z kolejnymi rundami, za wyjątkiem przedostatniej, gdzie następuje wzrost.

\begin{figure}[H]
	\centering
	\includegraphics[width=\textwidth]{Resources/RVsR/RVsRdecision.PNG}
	\caption{Wykres zwycięskich zagrań algorytmu losowego} 
	\label{fig:RVsRdecision}
\end{figure} 

\begin{figure}[H]
	\centering
	\includegraphics[width=\textwidth]{Resources/RVsR/RVsRguarddecision.PNG}
	\caption{Wykres szczegółowy zwycięskich zagrań karty Strażniczki algorytmu losowego} 
	\label{fig:RVsRguarddecision}
\end{figure}

Z dwóch powyższych wykresów (rys. \ref{fig:RVsRdecision} i rys. \ref{fig:RVsRguarddecision}) wynika, że zdecydowanie najczęstszym wygrywającym ruchem jest zagranie karty Baron. Około połowę mniej zwycięstw jest w wyniku zagrania karty Strażniczki. Warty uwagi jest fakt, że im bliżej końca gry tym częściej zwycięskim zagraniem jest zagranie karty Księcia na przeciwnika.

\clearpage
\section{Algorytm zachłanny versus Algorytm losowy}

\begin{figure}[H]
	\centering
	\includegraphics[width=\textwidth]{Resources/GVsR/GVsRwin.PNG}
	\caption{Wykres kołowy zwycięstw i remisów} 
	\label{fig:GVsRwin}
\end{figure}

Powyżej (rys. \ref{fig:GVsRwin}) widać zauważalną przewagę algorytmu zachłannego nad losowym.

\begin{figure}[H]
	\centering
	\includegraphics[width=\textwidth]{Resources/GVsR/GVsRroundwin.PNG}
	\caption{Wykres wygranych w danej rundzie} 
	\label{fig:GVsRroundwin}
\end{figure}

Na rys. \ref{fig:GVsRroundwin} widać różnicę w porównaniu do eksperymentu poprzedniego, gdzie obaj gracze grają losowo. Nie ma tak wyraźnego wzrostu liczby gier skończonych w przedostatniej rundzie. Wynika to ze znacznie mniejszej ilości gier zakończonych remisem.

\begin{figure}[H]
	\centering
	\includegraphics[width=\textwidth]{Resources/GVsR/GVsRdecision.PNG}
	\caption{Wykres zwycięskich zagrań algorytmu zachłannego} 
	\label{fig:GVsRdecision}
\end{figure} 

\begin{figure}[H]
	\centering
	\includegraphics[width=\textwidth]{Resources/GVsR/GVsRguarddecision.PNG}
	\caption{Wykres szczegółowy zwycięskich zagrań karty Strażniczki algorytmu zachłannego} 
	\label{fig:GVsRguarddecision}
\end{figure}
Wykres \ref{fig:GVsRdecision} pokazuje niewielkie zróżnicowanie zwycięskich zagrań, za wyjątkiem ostatniej tury. Wskazuje to na tendencję algorytmu zachłannego do wyeliminowania przeciwnika, kiedy tylko jest to możliwe.
Porównanie wykresów na rysunkach \ref{fig:RVsRguarddecision} oraz \ref{fig:GVsRguarddecision} pokazują ogromny wzrost znaczenia zagrania karty Strażniczki z wyborem karty Książę. Algorytm zachłanny osiąga zwycięstwo zagrywając Strażniczkę niemal tak samo często co przy zagraniu Barona.

\clearpage
\section{Algorytm minimaksowy versus Algorytm losowy}

\begin{figure}[H]
	\centering
	\includegraphics[width=\textwidth]{Resources/MmVsR/MmVsRwin.PNG}
	\caption{Wykres kołowy zwycięstw i remisów} 
	\label{fig:MmVsRwin}
\end{figure}

Powyższy wykres (rys. \ref{fig:MmVsRwin}) pokazuje dużą przewagę algorytmu minimaksowego nad algorytmem losowym.

\begin{figure}[H]
	\centering
	\includegraphics[width=\textwidth]{Resources/MmVsR/MmVsRroundwin.PNG}
	\caption{Wykres wygranych w danej rundzie} 
	\label{fig:MmVsRroundwin}
\end{figure}

Wykres na rys. \ref{fig:MmVsRroundwin} pokazuje wyraźną tendencję do spadku średniej liczby wygranych gier wraz z kolejnymi turami, co występuje również w przypadku algorytmu zachłannego.

\begin{figure}[H]
	\centering
	\includegraphics[width=\textwidth]{Resources/MmVsR/MmVsRdecision.PNG}
	\caption{Wykres zwycięskich zagrań algorytmu minimaksowego} 
	\label{fig:MmVsRdecision}
\end{figure} 

Wykres na rysunku \ref{fig:MmVsRdecision} pokazuje, że najskuteczniejszym zagraniem jest zagranie karty Baron. 

\begin{figure}[H]
	\centering
	\includegraphics[width=\textwidth]{Resources/MmVsR/MmVsRguarddecision.PNG}
	\caption{Wykres szczegółowy zwycięskich zagrań karty Strażniczki algorytmu minimaksowego} 
	\label{fig:MmVsRguarddecision}
\end{figure}

Na rys. \ref{fig:MmVsRguarddecision} widać, że zagranie karty Strażniczki daje najwięcej zwycięstw gdy jest zagrana z wyborem na kartę Księcia. Zdecydowanie mniej korzystny jest wybór karty Pokojówki i Barona.

\clearpage
\section{Algorytm MCTS versus Algorytm Losowy}
Dane z eksperymentów symulacyjnych, w których bierze udział algorytm MCTS, opierają się na próbie 1000 gier, ponieważ większa próba znacząco zwiększa czas oczekiwania na wyniki. Wynika to z charakteru algorytmu, który działa przez zadany czas. W przeprowadzonych symulacjach czas działania algorytmu ustawiony był na 100ms.

\begin{figure}[H]
	\centering
	\includegraphics[width=\textwidth]{Resources/MctsVsR/MctsVsRwin.PNG}
	\caption{Wykres kołowy zwycięstw i remisów} 
	\label{fig:MctsVsRwin}
\end{figure}

Powyższy wykres (rys. \ref{fig:MctsVsRwin}) pokazuje w tym wypadku zauważalną przewagę algorytmu losowego nad algorytmem Monte Carlo Tree Search, mimo, że był drugim graczem. Spowodowane jest to między innymi błędnie przyjętym założeniem, że podstawowa forma algorytmu MCTS osiągnie dobre wyniki mimo czynnika losowego, o ile ilość symulacji na turę będzie odpowiednio duża.

\begin{figure}[H]
	\centering
	\includegraphics[width=\textwidth]{Resources/MctsVsR/MctsVsRroundwin.PNG}
	\caption{Wykres wygranych w danej rundzie} 
	\label{fig:MctsVsRroundwin}
\end{figure}

Wykres na rys. \ref{fig:MctsVsRroundwin} pokazuje, że średnia liczba zwycięstw wzrasta w ostatnich turach, co jest charakterystyczne również dla eksperymentu z dwoma algorytmami losowymi.

\begin{figure}[H]
	\centering
	\includegraphics[width=\textwidth]{Resources/MctsVsR/MctsVsRdecision.PNG}
	\caption{Wykres zwycięskich zagrań algorytmu MCTS} 
	\label{fig:MctsVsRdecision}
\end{figure} 

\begin{figure}[H]
	\centering
	\includegraphics[width=\textwidth]{Resources/MctsVsR/MctsVsRguarddecision.PNG}
	\caption{Wykres szczegółowy zwycięskich zagrań karty Strażniczki algorytmu MCTS} 
	\label{fig:MctsVsRguarddecision}
\end{figure}

Wykresy na rysunkach \ref{fig:MctsVsRdecision} i \ref{fig:MctsVsRguarddecision} pokazują wyraźną tendecję algorytmu MCTS do zagrywania karty Strażniczki z wyborem na Kapłana.

\clearpage
\section{Algorytm Minimaksowy versus Algorytm Zachłanny}
\begin{figure}[H]
	\centering
	\includegraphics[width=\textwidth]{Resources/MirrorMmVsG/GVsMmWin.PNG}
	\caption{Wykres kołowy zwycięstw i remisów} 
	\label{fig:MirrorGVsMmWin}
\end{figure}

\begin{figure}[H]
	\centering
	\includegraphics[width=\textwidth]{Resources/MirrorMmVsG/GVsMmRoundWin.PNG}
	\caption{Wykres wygranych w danej rundzie} 
	\label{fig:MirrorGVsMmRoundWin}
\end{figure}

Powyższe wykresy (rys. \ref{fig:MirrorGVsMmWin} i \ref{fig:MirrorGVsMmRoundWin}) pokazują niewielką przewagę algorytmu minimaksowego nad zachłannym, który przeważa w rundach od 1 do 8. W rundach 9 i 10 przeważa algorytm zachłanny, z czego wynika, że częściej wygrywa gry kończące się porównaniem siły kart.

\begin{figure}[H]
	\centering
	\includegraphics[width=\textwidth]{Resources/MirrorMmVsG/MmVsGDecision.PNG}
	\caption{Wykres zwycięskich zagrań algorytmu minimaksowego} 
	\label{fig:MirrorMmVsGDecision}
\end{figure} 

\begin{figure}[H]
	\centering
	\includegraphics[width=\textwidth]{Resources/MirrorMmVsG/GVsMmDecision.PNG}
	\caption{Wykres zwycięskich zagrań algorytmu zachłannego} 
	\label{fig:MirrorGVsMmDecision}
\end{figure} 

\begin{figure}[H]
	\centering
	\includegraphics[width=\textwidth]{Resources/MirrorMmVsG/MmVsGGuardDecision.PNG}
	\caption{Wykres szczegółowy zwycięskich zagrań karty Strażniczki algorytmu minimaksowego} 
	\label{fig:MmVsGGuardDecisionn}
\end{figure}

\begin{figure}[H]
	\centering
	\includegraphics[width=\textwidth]{Resources/MirrorMmVsG/GVsMmGuardDecision.PNG}
	\caption{Wykres szczegółowy zwycięskich zagrań karty Strażniczki algorytmu zachłannego} 
	\label{fig:GVsMmGuardDecision}
\end{figure}


Z wykresów na rysunkach \ref{fig:MirrorMmVsGDecision}, \ref{fig:MirrorGVsMmDecision}, \ref{fig:MmVsGGuardDecisionn} i \ref{fig:GVsMmGuardDecision} wynika, że oba algorytmy wykazują duże podobieństwa - najczęściej wygrywają poprzez zagranie karty Barona, a zagrywając kartę Strażniczki najczęściej wybierają kartę Księcia.

\clearpage
\section{Algorytm Zachłanny versus Algorytm MCTS}

\begin{figure}[H]
	\centering
	\includegraphics[width=\textwidth]{Resources/MirrorMctsVG/GVsMctsWin.PNG}
	\caption{Wykres kołowy zwycięstw i remisów} 
	\label{fig:GVsMctsWin}
\end{figure}

\begin{figure}[H]
	\centering
	\includegraphics[width=\textwidth]{Resources/MirrorMctsVG/GVsMctsRoundWin.PNG}
	\caption{Wykres wygranych w danej rundzie} 
	\label{fig:GVsMctsRoundWin}
\end{figure}

Powyższe wykresy (rys. \ref{fig:GVsMctsWin} i \ref{fig:GVsMctsRoundWin}) pokazują bardzo wyraźną przewagę algorytmu zachłannego nad MCTS.

\begin{figure}[H]
	\centering
	\includegraphics[width=\textwidth]{Resources/MirrorMctsVG/MctsVsGDecision.PNG}
	\caption{Wykres zwycięskich zagrań algorytmu MCTS} 
	\label{fig:MctsVsGDecision}
\end{figure} 

\begin{figure}[H]
	\centering
	\includegraphics[width=\textwidth]{Resources/MirrorMctsVG/GVsMctsDecision.PNG}
	\caption{Wykres zwycięskich zagrań algorytmu zachłannego} 
	\label{fig:GVsMctsDecision}
\end{figure} 

Powyższe wykresy (rys. \ref{fig:MctsVsGDecision} i \ref{fig:GVsMctsDecision}) zwracają uwagę na nieregularność algorytmu MCTS do zwyciężania zagraniem karty Baron. Może to wynikać z tego, że MCTS częściej zatrzymuje w ręce kartę o wyższym numerze, i kiedy przeciwnik zagrywa kartę Barona, przegrywa.

\begin{figure}[H]
	\centering
	\includegraphics[width=\textwidth]{Resources/MirrorMctsVg/MctsVsGGuardDecision.PNG}
	\caption{Wykres szczegółowy zwycięskich zagrań karty Strażniczki algorytmu MCTS} 
	\label{fig:MctsVsGGuardDecision}
\end{figure}

\begin{figure}[H]
	\centering
	\includegraphics[width=\textwidth]{Resources/MirrorMctsVg/GVsMctsGuardDecision.PNG}
	\caption{Wykres szczegółowy zwycięskich zagrań karty Strażniczki algorytmu zachłannego} 
	\label{fig:GVsMctsGuardDecision}
\end{figure}

Na powyższych wykresach (rys. \ref{fig:MctsVsGGuardDecision} i \ref{fig:GVsMctsGuardDecision}) można zauważyć zupełnie inne tendencje do zagrywania karty Strażniczki. Algorytm MCTS niemal zawsze wybiera kartę Kapłana, natomiast algorytm zachłanny mimo że preferuje wybór Księcia, to niemal równie często wybiera inne typy kart.

\clearpage
\section{Algorytm Minimaksowy versus Algorytm MCTS}

\begin{figure}[H]
	\centering
	\includegraphics[width=\textwidth]{Resources/MirrorMmVsMcts/MmVsMctsWin.PNG}
	\caption{Wykres kołowy zwycięstw i remisów} 
	\label{fig:MmVsMctsWin}
\end{figure}

\begin{figure}[H]
	\centering
	\includegraphics[width=\textwidth]{Resources/MirrorMmVsMcts/MmVsMctsRoundWin.PNG}
	\caption{Wykres wygranych w danej rundzie} 
	\label{fig:MmVsMctsRoundWin}
\end{figure}

Powyższe wykresy (rys. \ref{fig:MmVsMctsWin} i \ref{fig:MmVsMctsRoundWin}) pokazują przewagę algorytmu minimaksowego nad algorytmem MCTS.

\begin{figure}[H]
	\centering
	\includegraphics[width=\textwidth]{Resources/MirrorMmVsMcts/MctsVsMmDecision.PNG}
	\caption{Wykres zwycięskich zagrań algorytmu MCTS} 
	\label{fig:MctsVsMmDecision}
\end{figure} 

\begin{figure}[H]
	\centering
	\includegraphics[width=\textwidth]{Resources/MirrorMmVsMcts/MmVsMctsDecision.PNG}
	\caption{Wykres zwycięskich zagrań algorytmu minimaksowego} 
	\label{fig:MmVsMctsDecision}
\end{figure} 

Na powyższych wykresach (rys. \ref{fig:MctsVsMmDecision} i  \ref{fig:MmVsMctsDecision}) warto zwrócić uwagę na statystykę zwycięskich zagrań w ostatnich rundach. Algorytm minimaksowy do końca zagrywa karty Barona i często wykorzystuje Księcia. Zwycięskie zagrania algorytmu MCTS w ostatnich rundach są bardziej równomiernie rozłożone, a w ostatniej rundzie nigdy nie zagrywa karty Barona.

\begin{figure}[H]
	\centering
	\includegraphics[width=\textwidth]{Resources/MirrorMmVsMcts/MctsVsMmGuardDecision.PNG}
	\caption{Wykres szczegółowy zwycięskich zagrań karty Strażniczki algorytmu MCTS} 
	\label{fig:MctsVsMmGuardDecision}
\end{figure}

\begin{figure}[H]
	\centering
	\includegraphics[width=\textwidth]{Resources/MirrorMmVsMcts/MmVsMctsGuardDecision.PNG}
	\caption{Wykres szczegółowy zwycięskich zagrań karty Strażniczki algorytmu minimaksowego} 
	\label{fig:MmVsMctsGuardDecision}
\end{figure}

Na powyższych wykresach (rys. \ref{fig:MctsVsMmGuardDecision} i \ref{fig:MmVsMctsGuardDecision}) widać, że algorytm MCTS zachowuje się podobnie jak wcześniej, natomiast algorytm minimaksowy zdecydowanie częściej zagrywa kartę Strażniczki z wyborem Księżniczki niż w było to w przypadku algorytmu zachłannego.

\section{Wnioski}
Na podstawie przeprowadzonych badań można sformułować następujące wnioski:
\begin{enumerate}
	\item Zdecydowana większość gier kończy się w pierwszych 3 rundach, głównie poprzez zagranie kart Strażniczki i Barona. Wynika z tego również, że gracz rozpoczynający ma znaczną przewagę nad drugim graczem.
	\item W kolejnych rundach średnia ilość zwycięstw maleje, a następnie rośnie w dwóch ostatnich. Wynika to z dwóch zjawisk: po pierwsze, przy końcu gry gracze wyzbyli się już kart ofensywnych, czyli Strażniczki, Barona i Księcia, wobec czego dochodzi do porównania sił kart. Po drugie, jeśli karty ofensywne pozostały w grze, znacznie wzrasta skuteczność ich użycia, co widać po częstości zagrań Strażniczki ze wskazaniem na Księżniczkę, bądź Księcia ze wskazaniem na przeciwnika (spodziewając się, że ma Księżniczkę).
	\item Ze statystyk można wywnioskować, że zasady gry ,,Love Letter'' promują ofensywne zagrania.
	\item Algorytm losowy nie wymaga głębokiej analizy - niemniej jednak warty uwagi jest fakt, że jest bardziej skuteczny niż algorytm MCTS.
	\item Algorytm zachłanny osiąga wysokie wyniki w porównaniu z innymi algorytmami i niewiele niższe niż algorytm minimaksowy. Wynika to ze wspomnianego faworyzowania przez grę zagrań, które jak najszybciej zakończą grę. 
	\item Algorytm minimaksowy osiąga wyniki tylko niewiele lepsze niż algorym zachłanny, pomimo znacznie dłuższego czasu działania. Warto pamiętać, że w przedstawionym wariancie dokonuje on tylko przeszukania drzewa rozwiązań do 1 ruchu w przód. Algorytm ten osiąga przewagę nad zachłannym w rundach środkowych, jednak różnica zwycięstw jest niewielka.
	\item Efektywność algorytmu Monte Carlo Tree Search jest zdecydowanie poniżej oczekiwań, co widać szczególnie po wynikach eksperymentu z algorytmem losowym. Przyczyną tego stanu rzeczy jest moje błędne założenie, że negatywny wpływ elementu losowego na algorytm może zostać zniwelowany przez ilość symulacji przeprowadzanych przez MCTS. Niestety, implementacja algorytmu w formie podstawowej sprawia, że wpada on w pewną pułapkę już podczas pierwszego kroku. Jest to spowodowane tym, że gracz w danym momencie gry nie zna całego jej stanu, lecz zbiór informacyjny $I_k$s, w związku z tym w korzeniu drzewa zawarty jest jeden wylosowany stan z tego zbioru informacyjnego. Biorąc pod uwagę ilość stanów w tym zbiorze, szansa, że zostanie wylosowany ten faktyczny, jest niewielka. Każda symulacja wykonana na błędnie założonym stanie początkowym w korzeniu drzewa algorytmu MCTS będzie prowadzić do błędnych wniosków. W efekcie MCTS osiąga gorsze wyniki niż algorytm losowy. Można by to podsumować potocznym stwierdzeniem, że ,,lepszy jest brak wiedzy, niż wiedza nieprawdziwa".
\end{enumerate}
