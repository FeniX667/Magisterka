\chapter{Rezultaty}
\label{cha:rozdz5}

Random vs Random
Random vs Greedy
Random vs Minimax
Random vs MCTS

Greedy vs Random

\section{Czas działania na 1000 partii}

\section{Statystyki zwycięstw}

\section{Zbieżność algorytmów podejmowania decyzji}

\section{Wnioski}
MCTS wpada w pułapkę, bo w jednym weźle trzyma tylko jeden stan z otrzymanej loterii, przez co wyciąga tak bardzo błędne wnioski, że nawet strategia losowa jest lepsza. Algorytm minmax jest nieco słabszy niż zachłanny, ponieważ jak widać z wykresów najczęściej wygrywającym ruchem jest baron, natomiast algorytm minimaxowy mając na ręce barona i pokojówkę zawsze zagra pokojówkę, oddalając tym samym od siebie szanse przegranej.