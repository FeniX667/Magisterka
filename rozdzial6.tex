\chapter{Podsumowanie}
\label{cha:rozdz6}

Celem mojej pracy była analiza i porównanie efektywności wybranych algorytmów w podejmowaniu decyzji w grze ,,Love Letter''. Podstawowym założeniem było nie przeszukiwanie całego drzewa rozwiązań, lecz jedynie najbliższy poziom, bądź posłużenie się heurystyką.

Opisałem zasady gry ,,Love Letter'', a następnie oszacowałem ilość możliwych rozwiązań. Na tej podstawie przedstawiłem ją jako problem optymalizacyjny, który stał się podstawą do porównania efektywności algorytmów. Model gry zapisałem w postaci gry ekstensywnej opierając się na Teorii Gier.

Do analizy i porównania wybrałem następujące algorytmy: losowy, zachłanny, minimaksowy oraz Monte Carlo Tree Search, który jest algorytmem heurystycznym. Sposób działania każdego z nich został opisany i podałem przykłady ich wykorzystania w praktyce. Następnie zaprezentowałem pomysł ich użycia do podejmowania decyzji w grze ,,Love Letter''.

Kolejno zaprezentowałem założenia programu, w którym zaimplementowane zostały zasady gry oraz wspomniane algorytmy. Przedstawiłem analizę wymagań oraz diagramy objaśniające strukturę programu. W analizie post implementacyjnej wskazałem główny problem związany z napisaniem programu, jakim okazał się niedostateczny poziom abstrakcji, znacznie zwiększający objętość kodu i tym samym ryzyko popełnienia błędu.

Następnie zaprezentowałem wyniki przeprowadzonych przeze mnie symulacji gier. W analizie skupiłem się na ilości zwycięstw danego algorytmu oraz najczęściej wygrywających zagraniach. Ważnym wnioskiem było wskazanie najskuteczniejszego z analizowanych algorytmów, jakim okazała się moja implementacja algorytmu minimaksowego. Z pewnością jego skuteczność mogłaby być wyższa, jednak z racji charakteru gry, który preferuje szybkie kończenie rund ponad zagrywki przedłużające grę, czas działania algorytmu rósł by zdecydowanie szybciej niż jego efektywność. 

Moim największym zaskoczeniem były wyniki algorytmu MCTS, które były gorsze nawet od wyników algorytmu losowego. Taki stan rzeczy wynika z błędnie przyjętych na początku założeń, że podstawowa wersja tego algorytmu będzie w stanie osiągać dobre wyniki w tej grze. Mimo to uważam, że po odpowiedniej modyfikacji polegającej na zamianie stanów znajdujących się w węzłach na zbiory informacyjne, byłby on w stanie osiągać wyniki równe, a nawet lepsze, od algorytmu minimaksowego.