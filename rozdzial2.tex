\chapter{Gra karciana Love Letter}
\label{cha:rozdz2}

W tym rozdziale opisuję kontekst oraz zasady gry Love Letter. Tłumaczę działanie każdej karty oraz przedstawiam główny cel gry - wygranie określonej ilości rund. Następnie przedstawiam problem i analizuję jego złożoność. Wszystkie załączone zdjęcia oraz instrukcja zaczerpnięte są z [\ref{bib:loveLetterGame}] oraz [\ref{bib:loveLetterWebsite}].

\begin{figure}[h]
	\centering
	\includegraphics{Resources/ll_main_image.png}
	\caption{Love Letter - okładka} 
	\label{fig:llMainImage}
\end{figure}

\section{Opis zasad gry}
\label{sec:opisGry}
W trakcie gry wcielamy się w rolę jednego z adoratorów księżniczki starającego się o zdobycie jej serca. W tym celu przygotowaliśmy list miłosny, który chcemy jej dostarczyć. Niestety, księżniczka pogrążona jest obecnie w żałobie i nie przyjmuje do siebie nikogo obcego, w związku z czym musimy znaleźć inny sposób na przekazanie jej naszego listu. Oprócz księżniczki, na dworze znajdują się inne postacie, z których każda ma mniejszy lub większy dostęp do komnat naszej wybranki i może oddać jej list. Przekazujemy więc naszą przesyłkę swojemu tajnemu posłańcowi, a na koniec gry księżniczka jako pierwszy przeczyta ten list, który został przekazany przez najbardziej zaufaną postać. Serce wybranki zdobywa gracz, który jako pierwszy przekaże w ten sposób od 4 do 7 listów, w zależności od liczby graczy.

\section*{Cel i ustawienie początkowe}
\label{sec:celIUstawieniePoczatkowe}
Love Letter rozgrywa się jako serię rund. Grę wygrywa gracz o następującej ilości wygranych rund:
\begin{itemize}
	\item 7 w grze na 2 graczy,
	\item 5 w grze na 3 graczy,
	\item 4 w grze na 4 graczy.
\end{itemize}
Każda runda dzieli się na tury, w których naprzemiennie jeden z graczy wykonuje ruch. Grę wygrywa ten z nich, który na końcu ostatniej tury posiada kartę o wyższym numerze.

Ustawienie początkowe każdej rundy wygląda następująco:
\begin{itemize}
	\item przetasuj karty
	\item odrzuć 1 wierzchnią kartę nie odkrywając jej (nie bierze udziału w rundzie),
	\item jeśli gra tylko 2 graczy, odrzuć 3 wierzchnie karty, odkryte,
	\item rozdaj po 1 karcie wszystkim graczom,
	\item jeśli jest to pierwsza runda, grę zaczyna gracz, który jako ostatni był na randce, w przeciwnym wypadku zwycięzca poprzedniej rundy.
\end{itemize}

\section*{Tura gracza i opis kart}
\label{sec:turaGracza}
Podczas swojej tury gracz dociąga jedna kartę ze stosu. Następnie wybiera jedną z dwóch kart, które posiada już w ręce, kładzie ją przed sobą tak, by była widoczna dla wszystkich i zastosowuje opisany na niej efekt - nawet jeśli jest negatywny. Zagrana karta pozostaje odkryta przez całą rundę, a druga pozostaje w ręce. Następnie tura przechodzi na osobę po lewej stronie aktywnego gracza.

W grze znajduje się 16 kart, w 8 typach. Każda z typów kart posiada wartość od 1 do 8. Są to kolejno: 4 karty Strażniczki, po 2 karty Kapłana, Barona, Pokojówki i Księcia, oraz po jednej karcie Króla, Hrabiny i Księżniczki. Ich szczegółowy opis wraz z wyglądem znajduje się poniżej:

\clearpage
\begin{figure}[h]
	\centering
	\includegraphics[scale=0.5]{Resources/Love_Letter_Card_Guard.png}
	\caption{Strażniczka} \label{fig:Love_Letter_Card_Guard}
\end{figure}
Na rysunku \ref{fig:Love_Letter_Card_Guard} przedstawiona jest karta typu Strażniczka. Zagrywając tę kartę należy wskazać jednego z pozostałych graczy i odgadnąć kartę którą posiada. Jeśli karta została prawidłowo odgadnięta, wskazany gracz odrzuca ją i przegrywa rundę.

\begin{figure}[h]
	\centering
	\includegraphics[scale=0.5]{Resources/Love_Letter_Card_Priest.png}
	\caption{Kapłan} \label{fig:Love_Letter_Card_Priest}
\end{figure}
Rysunek \ref{fig:Love_Letter_Card_Priest} przedstawia kartę typu Kapłan. Zagrywając tę kartę należy podglądnąć kartę wybranego gracza.

\clearpage
\begin{figure}[h]
	\centering
	\includegraphics[scale=0.5]{Resources/Love_Letter_Card_Baron.png}
	\caption{Baron} \label{fig:Love_Letter_Card_Baron}
\end{figure}
Na rysunku \ref{fig:Love_Letter_Card_Baron} przedstawiona jest karta typu Baron. Po zagraniu tej karty należy w ukryciu porównać drugą posiadaną kartą z wybranym graczem. Następnie ten gracz, który ma kartę o mniejszej wartości odrzuca swoją kartę i przegrywa rundę. W przypadku remisu nic się nie dzieje.

\begin{figure}[h]
	\centering
	\includegraphics{Resources/Love_Letter_Card_Handmaid.png}
	\caption{Pokojówka} \label{fig:Love_Letter_Card_Handmaid}
\end{figure}
Rysunek \ref{fig:Love_Letter_Card_Handmaid} przedstawia kartę typu Pokojówka. Zagranie tej karty sprawia, że gracz jest niewrażliwy na efekt pozostałych kart do czasu swojej następnej tury.

\clearpage
\begin{figure}[h]
	\centering
	\includegraphics[scale=0.5]{Resources/Love_Letter_Card_Prince.png}
	\caption{Książe} \label{fig:Love_Letter_Card_Prince}
\end{figure}
Na rysunku \ref{fig:Love_Letter_Card_Prince} przedstawiona jest karta typu Książę. Zagranie pozwala wybrać dowolnego gracza (w tym siebie), zmusić go do odrzucenia posiadanej karty i pociągnięcia następnej.

\begin{figure}[h]
	\centering
	\includegraphics[scale=0.5]{Resources/Love_Letter_Card_King.png}
	\caption{Król} \label{fig:Love_Letter_Card_King}
\end{figure}
Rysunek \ref{fig:Love_Letter_Card_King} przedstawia kartę typu Król. Po jej zagraniu należy wymienić się pozostałą kartą z innym graczem.

\clearpage
\begin{figure}[h]
	\centering
	\includegraphics{Resources/Love_Letter_Card_Countess.png}
	\caption{Hrabina} \label{fig:Love_Letter_Card_Countess}
\end{figure}
Rysunek \ref{fig:Love_Letter_Card_Countess} przedstawia kartę typu Hrabina. Ta karta ma działanie pasywne. Nie wywiera efektu po zagraniu, natomiast zmusza gracza do jej zagrania jeśli równocześnie posiada na ręce kartę typu Książę lub Król.

\begin{figure}[h]
	\centering
	\includegraphics[scale=0.5]{Resources/Love_Letter_Card_Princess.png}
	\caption{Księżniczka} \label{fig:Love_Letter_Card_Princess}
\end{figure}
Na rysunku \ref{fig:Love_Letter_Card_Princess} przedstawiona jest karta typu Księżniczka. Zagranie tej karty oznacza natychmiastową przegraną w rundzie. Ta zasada działa również, gdy gracz został zmuszony do zagrania tej karty, np. przez efekt karty Książe.
\clearpage

\section{Definicja i analiza problemu}
\label{sec:opisProblemu}
Z wyżej przedstawionych zasad wynika, że gra cechuje się wysokim stopniem losowości i jest niedeterministyczna, więc można ją przedstawić jako problem optymalizacyjny$^{[\ref{bib:wiki_ProblemOptymalizacyjny}]}$. W związku z tym powstaje również pytanie, jaka jest najlepsza strategia mieszana$^{[\ref{bib:wiki_StrategiaTeoriaGier}]}$ maksymalizująca prawdopodobieństwo wygrania gry. Dla dalszych rozważań zakładam, że w gra toczy się pomiędzy dwoma graczami.

Najprostszym sposobem na znalezienie takiej strategii, byłoby stworzenie drzew probabilistycznych dla wszystkich możliwych stanów początkowych gry a następnie opracowanie algorytmu podejmowania decyzji opartego na danych statystycznych. Jednakże ilość i rozmiar tych drzew może być zbyt duża by znaleźć rozwiązanie problemu w rozsądnym czasie. Z tego powodu postanowiłem najpierw oszacować ile jest wszystkich możliwych przebiegów gry. Nim przejdę do obliczeń, wprowadzę kilka definicji by ustandaryzować używane pojęcia:
\begin{itemize}
	\item \textit{Decyzja} - inaczej \textit{Zagranie}, jest to typ karty wraz ze sposobem jej wykorzystania. Przykładowo: Zagranie karty typu Strażniczka z wyborem karty typu Król, lub zagranie karty typu Książę z wyborem na gracza przeciwnego.
	\item \textit{Podjęcie decyzji} - to wybór zagrania, które zostanie użyte jako ruch w grze. Podjęcie decyzji to inaczej zwrócenie zagrania przez algorytm.
	\item \textit{Strategia} - algorytm, który podejmuje decyzje. 
	\item \textit{Scenariusz} - inaczej przebieg gry, chronologiczny spis decyzji podjętych przez obu graczy od początku do końca rundy; Jedna ze ścieżek w drzewie probabilistycznym dla danego stanu początkowego rundy.
\end{itemize}

\subsection*{Analiza złożoności}
Na przebieg każdej rundy wpływ mają następujące czynniki:
\begin{itemize}
	\item Kolejność kart w talii na początku rundy
	\item Zagrywanie kart przez graczy
\end{itemize}
Zacząłem od oszacowania ilości możliwych stanów początkowych. W każdej rundzie bierze udział wszystkie 16 kart. Zakładając, że każda z nich jest unikalna, to  liczba wszystkich możliwych kolejności kart to permutacja, którą obliczam wzorem podanym w [\ref{bib:tabliceMatematyczne}]:

\begin{center}
	$P_n = n!$ , gdzie $n\in N^+$
\end{center}

Dla  $n$ = 16, $n!=20 922 789 888 000$. Część kart się powtarza, więc tę liczbę należy jeszcze podzielić przez permutacje powtarzających się kart Strażniczki, Kapłana, Barona, Pokojówki oraz Księcia. Razem jest to $4! * 2! * 2! * 2! * 2! =  384$. Ostatecznie wynika, że liczba unikalnych kolejności kart wynosi: 

\begin{center}
	$20 922 789 888 000 / 32 = 54486432000$ - 54mld, 864mln i 432 tys.
\end{center}

Nie jest to jednak liczba wszystkich dostępnych scenariuszy. W każdej turze gracz ma do wyboru co najmniej dwa zagrania, ponieważ tyle ma dostępnych kart. Jednakże, w przypadku karty Strażniczki możliwości jest więcej, ponieważ można wytypować 7 typów kart, a w jednym przypadku może to pokonać przeciwnika i skończyć rundę. W związku z tym, by oszacować liczbę scenariuszy na zadanej kolejności kart, posłużyłem się następującą metodą:
\begin{itemize}
	\item zgodnie z zasadami gry dla dwóch graczy, odrzucam łącznie 4 pierwsze karty (1 zakryta, 3 odkryte).
	\item rozdaję po 1 karcie obu graczom. Pozostaje 10 kart w talii.
	\item zakładając, że żadna decyzja nie spowoduje przerwania rundy, gracze łącznie 10 razy pociągną kartę, więc podejmą 10 decyzji.
	\item każdą decyzję można przedstawić jako 0 (zagranie posiadanej karty) lub 1 (zagranie pociągniętej karty) - w tym miejscu jeśli istnieje możliwość zagrania karty w wieloraki sposób, upraszczam to do jednego zagrania niekończącego rundę.
\end{itemize}
Na podstawie powyższego można oszacować, że możliwych scenariuszy dla danej kolejności kart jest $2^{10}=1024$. Łącząc tę liczbę z ilością możliwych kolejności kart, utrzymujemy przybliżoną liczbę scenariuszy:

\begin{center}
	$54486432000 * 1024 = 55794106368000 \approx  5.8*10^{12}$
\end{center}

Z uwagi na rząd wielkości, stworzenie strategii na podstawie analizy statystycznej wszystkich dostępnych scenariuszy jest problemem \textit{NP-trudnym}. Z tego powodu, zamiast odpowiadać na pytanie ,,Jaka jest najlepsza strategia podejmowania decyzji?'', dużo łatwiej będzie odpowiedzieć na pytanie ,,która z podanych strategii jest najlepsza?'', gdyż jest to charakterystyczna cecha problemów klasy \textit{NP}. Kierując się tą zasadą, w następnym rozdziale opisałem wybrane strategie, których skuteczność sprawdzę implementując je w napisanej przeze mnie aplikacji. Pozostaje jeszcze sformalizować przedstawiony problem za pomocą modelu matematycznego.


\section{Model matematyczny problemu}
Patrząc z perspektywy jednego z graczy, cała gra składa się z szeregu etapów (tur), gdzie w każdym etapie gracz podejmuje decyzję, a stan początkowy następnego etapu jest wynikiem podjętej decyzji oraz reakcji gracza przeciwnego. Takie zachowanie można przedstawić jako wieloetapowy proces podejmowania decyzji w warunkach niepewności$^{[\ref{bib:wiki_TeoriaDecyzji}]}$, który ogólnie zdefiniowałem następująco: 

\clearpage
\begin{figure}[h]  % TODO poprawić by xn było xi na obrazku, bo i to |X|
	\centering
	\includegraphics[]{Resources/Schemat_Modelu.png}
	\caption{Schemat modelu} \label{fig:Schemat_Modelu}
\end{figure}
\begin{itemize}
	\item n - etap gry
	\item $s_{n-1} \in S_{n-1}$ - stan wejściowy
	\item $x_i \in X_n$ - podjęta decyzja
	\item $s_{n} \in S_n$ - stan wyjściowy
	\item $X_n$ - zbiór dopuszczalnych decyzji dla etapu n-tego
	\item $S_n$ - zbiór dopuszczalnych stanów dla etapu n-tego
	\item $T_n: \{x_i, s_{n-1} \} \rightarrow \{s_n, X_n\}$ - funkcja przejścia
	\item $Q = max(P_w)$ - funkcja celu, gdzie $P_w$ oznacza prawdopodobieństwo wygranej
	\item $D: \{X_n, s_{n-1}\} \rightarrow x_n$ - strategia podejmowania decyzji
\end{itemize}
Funkcja przejścia $T_n$ wynika bezpośrednio z opisanych wcześniej zasad oraz zawiera reakcję gracza przeciwnego, w efekcie zwracając jeden z możliwych stanów należących do $S_n$ oraz zbiór dopuszczalnych decyzji $X_n$. Również wszystkie $S_0 .. S_n$ oraz $X_0 .. X_n$ wynikają z zasad gry. Jedynym nieokreślonym elementem pozostaje $D$, a moje propozycje na jego zdefiniowanie przedstawiam w następnym rozdziale.